\documentclass{report}
\usepackage[margin=2.5cm]{geometry}
\usepackage{amssymb}
\usepackage{color}

\newcommand{\smallz}{{\scriptscriptstyle Z}}
\newcommand{\smallw}{{\scriptscriptstyle W}}
\newcommand{\mz}{m_\smallz}
\newcommand{\mw}{m_\smallw}
\newcommand{\qsh}{Q_{sh}}
\newcommand{\ptw}{p_\perp^{\ell\nu}}
\newcommand{\ptz}{p_\perp^{\ell^+\ell^-}}
\newcommand{\ptv}{p_\perp^{\smallv}}
\newcommand{\amcnlo}{{\sc MG5\_aMC}\,\,}
\newcommand{\amcnlonospace}{{\sc MG5\_aMC}}
\newcommand{\powheg}{{\sc POWHEG}\,\,}
\newcommand{\powhegnospace}{{\sc POWHEG}}
\newcommand{\pythia}{{\sc Pythia8}\,\,}


\begin{document}

\noindent

\begin{center}
  \LARGE \textbf{Author answer to the referee report of JHEP\_187P\_0318}
\end{center}
\vspace{0.5cm}
We thank the referee for the positive report and for her/his insightful comments and detailed comments. Below, we address all the questions and suggestions contained in the report in order. To ease the reading, the points raised by the referee (quoted in italic) are followed by an  explanation of the actions undertaken to address them. 

\vspace{0.5cm}
\begin{enumerate}

    \item \emph{ At the beginning of Sec. 2 "LO" is defined as the three possible coupling
combinationas $\alpha^6$, $\alpha_s \alpha^5$, $\alpha_s^2 \alpha^4$. However for most of the paper "LO" is more 
and more taken equivalent to only mean the EW $\alpha^6$ contribution. (At first it still says "LO at $\alpha^6$", 
then it becomes
LO i.e. $\alpha^6$, and eventually in the LO+PS section the coupling order is dropped completely. 
This is of course just semantics, but it was quite confusing on the first reading and it would be good to make this a bit clearer from the start.}

Answer: Before section 3, ``if not specified explicitly LO refers to the order alpha6.
In the same way, NLO refers if not specified to the order alphas alpha6''.

    \item \emph{The ordering of the various programs in tables 1, 3, and 5 is very helpful.
I can perhaps understand the reasoning behind choosing this alphabetical
ordering, but I think it is misplaced here and makes it quite hard to get
anything out of this. It would be much more useful to have programs with similar levels of approximations next to each other. So keep Phantom and Whizard together (both having the same approxmations), then Bonsay, Powheg, VBFNLO (all without interference with VBFNLO adding the s-channel), then MG5\_AMC (also adding interference), and last MoCaNLO+Recola having the full result.}

Answer: We have changed the order as suggested.
For consistency, we have also changed the order of the description of the various codes in section...

\item \emph{In table 3: Please indicate which ones are exact at LO, and which ones are not. Also, regarding the discussion in Sec. 4.3 and the fact that several full LO
predictions differ far outside their statistical MC uncertainties. This I find
extremely puzzling and the offered explanations are quite unsatisfactory. After
all, at LO, MC statistics is just statistics, I don't understand how statistical
uncertainties can be aggressive or not, they are what they are.
Also, the complex-mass scheme is a well-defined procedure, so how can different implementations of it lead to numerical differences? (If there are somewhat different approximations being employed, it would be good to spell this out clearly.) The fact that the differences are at the 0.5\% level
cannot hide the fact that they are much much bigger than the MC statistics.
Taken at face value, that means that there are clear systematic differences where from what I understand there shouldn't be any, so the reader is left to wonder what is going on? To play the devil's advocate, if there are systematic differences at LO that cannot be understood, how can I trust the NLO comparison?}

Answer: PSP and maybe then just argue about the physical message.

\item \emph{ page 16, 2nd column, line 13, typo: "elment" $\to$ "element"}\\
    
Answer: It has been corrected

\item \emph{ page 18, 2nd column, line 48: "on the other ..." the part of the sentence is
    nonsensical.}
    
Answer: ``which clearly suggest not to rely on a single tool/parton shower, and on the other make'' ->
    ``which on the one hand clearly suggest not to rely on a single tool/parton shower, and on the other make''
    
\item \emph{page 19, 1st column, line 14:
I think it is important to not mistake "more realistic" for "conservative". The
scale variations are clearly underestimating the actual theoretical uncertainties. This does not mean that another method, which implies larger uncertainties, is "conservative", it is simply (hopefully) "more realistic". Conservative is easily (mis)interpreted as possibly overestimated, which I don't think is the case here. (In fact, there is no guarantee that even the comparison of different tools will provide an estimate that covers the next order, so calling this a conservative estimate is really misplaced.}

Answer: "conservative" -> "more realistic" 

\end{enumerate}

\end{document}

%%% Local Variables:
%%% mode: latex
%%% TeX-master: t
%%% End:
