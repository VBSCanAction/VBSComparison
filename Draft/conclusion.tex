In the present article, a detailed study of the process $\Pp\Pp\to\mu^+\nu_\mu{\rm e}^+\nu_{\rm e}\,\Pj\Pj+\mathrm{X}$ at the LHC has been presented, 
mainly focused on the electroweak production mechanism which involves the scattering of massive vector-bosons.
Until very recently, when the complete calculation became available for the NLO QCD corrections (order $\mathcal O (\alphas\alpha^6)$),
the so-called VBS approximation was the standard for this kind of simulations. For this reason, various theoretical predictions 
have been compared to the full computation, both in a typical VBS fiducial region and also in more inclusive phase-space.
We have precisely quantified the differences that arise for several physical observables, 
in particular for the di-jet invariant mass and the rapidity-separation of the leading two jets.
This is the first time that such an in-depth study is performed.
%%%%and should be representative of the quality of the approximations for other VBS signatures.
Besides the study of fixed-order predictions, we have also investigated the impact of parton showering.
To that end, several LO and NLO event generators 
which are able to perform matching to parton showers have been employed, and various observables have been thoroughly compared.
While in general observables which are described at NLO accuracy show reasonable agreement among the tools, larger differences can 
appear for those observables described at a lower accuracy, such as those that involve the third jet.
In particular such differences are quite prominent in the central-rapidity region, and are the largest for those simulations which employ {\sc Pythia8}.
The effect
has been understood, and it can be partially mitigated by changing the recoil scheme of {\sc Pythia8} to distribute momenta within initial-final colour connections. These findings make it worth 
to further investigate these issues not only in the theoretical community, but also by experimental collaborations, for example by 
measuring related observables for similar processes.

%The results presented here are exclusively theoretical.
%Nonetheless, they should raise significant interests in the experimental collaborations.
The last part of our work is devoted to 
 remarks and recommendations concerning the usage of theoretical predictions by experimental collaborators.
\begin{itemize}
    \item As found in Ref.~\cite{Biedermann:2017bss}, the NLO EW corrections of order $\mathcal{O}{\left(\alpha^{7}\right)}$ are 
        the dominant NLO contribution to the process $\Pp\Pp\to\mu^+\nu_\mu{\rm e}^+\nu_{\rm e}\,\Pj\Pj+\mathrm{X}$.
        It is thus highly desirable to combine them with NLO-QCD predictions matched with parton shower, or at least that they are accounted for
        by experimental analyses. Since, as shown in Ref.~\cite{Biedermann:2016yds}, 
        these large EW corrections originate from the Sudakov logarithms which factorise, we recommend to combine them with QCD 
        corrections in a multiplicative way. The estimate of missing higher-order EW corrections can be done, 
        in a first approximation, by considering $\pm \delta^2_{\rm NLO EW}$,\footnote{The quantity $\delta_{\rm NLO EW}$ through the relation $\sigma_{\rm NLO EW} = \sigma_{\rm LO} \left(1+ \delta_{\rm NLO EW}\right).$} while the missing higher-order mixed QCD-EW corrections 
        can be estimated by taking the difference between the multiplicative and additive prescriptions.
        For more detailed studies of the combination of QCD and EW higher-order corrections, see
        \emph{e.g.}\, Ref.~\cite{Czakon:2017wor} in the context of top-pair production, or Ref.~\cite{Lindert:2017olm} for SM 
        backgrounds for dark matter searches at the LHC.
    \item For the typical fiducial region used by experimental collaborations for their measurements, 
        the agreement between the approximations and the full calculation is certainly satisfactory given 
        the current experimental precision, as well as the one foreseen for the near future \cite{CMSCollaboration:2015zni,CMS:2016rcn}.
        Nonetheless, care has to be taken when using such approximations, in particular if more inclusive phase-space cuts are used.

%{\bf MZ there are many repetitions here} \BJ{I would suggest to remove the following paragraph as it repeats what was already said before.}
%Concerning predictions that include parton shower, we have shown that for observables defined at NLO, the differences are reasonable.
%On the other hand, for observables only defined at NLO-QCD, such as the ones related to the third hardest jet, large differences can arise between different parton-shower algorithms.
%Therefore, jet-veto procedures or selections based on extra radiation should be avoided as they carry rather large uncertainties.

    \item In addition to the standard interpretation of EW signal versus QCD background, 
        combined measurements should also be presented as they are better defined theoretically. In fact, while at LO 
        the interference term can be included in the background component, at NLO the separation of EW and QCD components become more blurred, as, \emph{e.g.}\,
        at the order $\mathcal{O}{\left(\alphas\alpha^{6}\right)}$ both types of amplitudes contribute.
        Therefore, a combined measurement including the EW, QCD, and interference contributions is desirable.
        Note that with such a measurement a comparison to the SM would
        be straigtforward and still be sensitive to the EW component.
        In addition the QCD component could be subtracted based on a
        well-defined Monte-Carlo prediction.

    \item Since the inclusion of NLO QCD corrections gives a better control of extra QCD radiations and reduces the ambiguities related to the 
        matching details and/or the parton shower employed, we encourage the use of NLO-accurate event generators in experimental analyses. In doing
        so, special care should be employed in order to estimate the theoretical uncertainties, as the standard prescription based on 
        renormalisation and factorisation-scale variation is clearly inadequate. Rather, different generators and/or parton showers should be employed.

    \item The present study has focused on the orders $\mathcal{O}{\left(\alpha^{6}\right)}$ at 
    LO and $\mathcal{O}{\left(\alphas\alpha^{6}\right)}$ at NLO. NLO computations and publicly-available tools also exist for the QCD-induced process~\cite{Rauch:2016pai,Melia:2010bm,Melia:2011gk,Campanario:2013gea,Baglio:2014uba,Biedermann:2017bss,Alwall:2014hca}.

    \item For practical reasons, we have focused on the ${\rm W}^+{\rm W}^+$ signature. Nonetheless, 
    the observed features 
    (\emph{e.g.}\ validity of the VBS approximation or comparison of theoretical predictions matched to parton shower) should 
    be qualitatively similar for other VBS signatures with massive gauge bosons. For these other signatures, similar quantitative studies should be performed.
\end{itemize}
