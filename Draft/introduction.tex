Vector-boson scattering (VBS) is a class of processes that allow to probe the nature of Higgs-vector-vector couplings and quartic gauge couplings.
It is usually understood that VBS refers to the scattering of massive vector-bosons ($\PW^\pm,\PZ$), which therefore couple to the Higgs boson and can be longitudinally polarized.
The scattering of longitudinally polarized bosons is of particular interest, because the corresponding matrix elements feature both gauge and unitarity cancellations that strongly depend on the actual form of the Higgs sector.
A detailed study of this class of processes will therefore further constrain the Higgs couplings and hint at or exclude non-Standard Model Higgs bosons.

The process with two positively charged W bosons is the VBS process with the largest signal-to-background ratio at the LHC, for which evidence has been found in $8 \TeV$ data~\cite{Aad:2014zda,Khachatryan:2014sta} and which is now started to be observed~\cite{Sirunyan:2017ret} and measured~\cite{Aaboud:2016ffv} in $13\TeV$ data.
For now the measurements of VBS processes are limited by statistics but the situation will change in a near future.
On the theoretical side, it is thus of prime importance to provide precise predictions and infer their related systematic errors.

The $\PW^+\PW^+$ scattering is the simplest VBS process to calculate, because of the double-charge structure of the leptonic final state that limits the number of partonic processes and total number of Feynman diagrams for each process.
It also implies that irreducible backgrounds are comparatively small, which make this VBS process experimentally favourable in comparison to \eg to $\PW^+\PW^-$ scattering, which has the largest cross section.
Therefore, the $\PW^+\PW^+$ scattering is an ideal candidate for a detailed study of various theoretical predictions.

In the last few years, several leading order (LO) and next-to-leading order (NLO) computations became available for both the VBS process~\cite{Jager:2009xx,Jager:2011ms,Denner:2012dz,Rauch:2016pai} and its QCD-induced irreducible background process~\cite{Rauch:2016pai,Melia:2010bm,Melia:2011gk,Campanario:2013gea,Baglio:2014uba}.
These computations all rely on approximations, while recently the complete NLO corrections have been performed~\cite{Biedermann:2017bss}.
It is therefore interesting to infer in details the quality of the various approximations.
Indeed, apart from Ref.~\cite{Biedermann:2017bss} where it is commented on, \MP{more references?} no detailed comparison of the VBS approximations have been carried out.
Preliminary results have already been made public in Ref.~\cite{Anders:2018gfr}.

The hadronic process is $\Pp\Pp\to\mu^+\nu_\mu{\rm e}^+\nu_{\rm e}\,\Pj\Pj+\mathrm{X}$, which includes the $\PW^+\PW^+$ scattering.
This final state possesses three contributions at LO whose coupling orders are $\mathcal{O}{\left(\alpha^{6}\right)}$, $\mathcal{O}{\left(\alpha_{\rm s}\alpha^{5}\right)}$, and $\mathcal{O}{\left(\alphas^{2}\alpha^{4}\right)}$.
They are commonly referred to as electroweak (EW), interference, and QCD contributions, respectively.%
\footnote{The EW contribution is sometimes referred to as the VBS contribution even it possesses non-VBS contributions.}
Therefore, the present work starts with a LO study of these three contributions as a function of typical VBS cuts.
This allows to quantify the various contributions to the final state $\mu^+\nu_\mu{\rm e}^+\nu_{\rm e}\,\Pj\Pj$.
This is followed by a LO comparison between the various predictions at the level of the cross section and differential distributions.

At NLO, the process possesses four contributions of orders $\mathcal{O}{\left(\alpha^{7}\right)}$, $\mathcal{O}{\left(\alphas\alpha^{6}\right)}$, $\mathcal{O}{\left(\alphas^{2}\alpha^{5}\right)}$, and $\mathcal{O}{\left(\alphas^{3}\alpha^{4}\right)}$.
The largest one are the EW corrections~\cite{Biedermann:2017bss,Biedermann:2016yds} of order $\mathcal{O}{\left(\alpha^{7}\right)}$.
The contribution to the order $\mathcal{O}{\left(\alphas\alpha^{6}\right)}$ is the second largest NLO contribution and is often referred to as the QCD corrections to the VBS process.
In the following, this order is the one where our comparisons are focused on.
In this article we will refer to it as simply \emph{NLO}.
As for the LO study, the various predictions are compared at the level of the cross section and differential distributions.
In particular, this allows to infer the accuracy of the so-called VBS approximation, which we will define in more detail later.
To our knowledge, such a detailed study was still missing.

Finally, several predictions featuring parton shower are compared.
This allows to infer systematic differences between the various predictions.
This is the first time in the literature that such an analysis has been carried out \MP{True?}.

Obviously all VBS processes deserve such a detailed study but the present article sets standards for inferring systematics related to NLO corrections and beyond.


The article is organised as follow:
in Sec.~\ref{sec:definition} the studied process is defined.
Various approximation at LO and NLO are described in Sec.~\ref{sec:details}.
This is followed by a presentation of the programs used for the computations.
Sections~\ref{sec:LO} and \ref{sec:NLO} are devoted to a LO and NLO study, respectively.
Section~\ref{sec:matching} deals with the matching to parton shower.
The last section consists in concluding remarks and recommendations for experimental collaborations.
