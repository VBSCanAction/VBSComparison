The vector-boson scattering (VBS) process with two positively charged W boson is just starting to be measured at the LHC~\cite{Aad:2014zda,Aaboud:2016ffv,Khachatryan:2014sta}.
A whole class of new processes will therefore be measured during Run~II at the Large Hadron Collider (LHC).
For now the measurements of VBS processes are limited by statistics but the situation will change in a near future.
On the theoretical side, it is thus of prime importance to provide precise predictions and infer their related systematics.

The $\PW^+\PW^+$ scattering is probably the simplest VBS processes to compute and to measure due to its particular charge structure and low background, respectively.
Therefore, it is an ideal example for a detailed study of various theoretical predictions.
In the last few years, several (N)LO computations become available for both the VBS process~\cite{Jager:2009xx,Jager:2011ms,Denner:2012dz,Rauch:2016pai} and its QCD-induced irreducible background
process~\cite{Melia:2010bm,Melia:2011gk,Campanario:2013gea,Baglio:2014uba,Rauch:2016pai}.
These computations all relied on approximations while recently the complete NLO corrections became available \cite{Biedermann:2017bss}.
It is therefore interesting to infer in details the quality of the various approximations.
Indeed, apart from Ref.~\cite{Biedermann:2017bss} \MP{more references?} no detailed comparison of the VBS approximations have been carried out.

The hadronic process is ${\rm p}{\rm p}\to\mu^+\nu_\mu{\rm e}^+\nu_{\rm e}{\rm j}{\rm j}$ and it posses three contributions at leading order (LO) 
[$\mathcal{O}{\left(\alpha^{6}\right)}$, $\mathcal{O}{\left(\alphas^{2}\alpha^{4}\right)}$, and $\mathcal{O}{\left(\alpha_{\rm s}\alpha^{5}\right)}$].
They are refer to as EW, interference, and QCD contributions respectively.
The EW contribution is sometimes referred to as the VBS contribution even it possesses not-VBS contributions.
Therefore, we start with a LO study of these contributions as a function of typical VBS cuts.
This allows to understand the various contributions to the final state $\mu^+\nu_\mu{\rm e}^+\nu_{\rm e}{\rm j}{\rm j}$.
This is followed by a LO comparison between the various predictions at the level of the cross section and differential distributions.

At NLO, the process possesses four contributions of orders 
$\mathcal{O}{\left(\alpha^{7}\right)}$,
$\mathcal{O}{\left(\alphas\alpha^{6}\right)}$,
$\mathcal{O}{\left(\alphas^{2}\alpha^{5}\right)}$, and
$\mathcal{O}{\left(\alphas^{3}\alpha^{4}\right)}$.
The largest one is the EW corrections of order $\mathcal{O}{\left(\alpha^{7}\right)}$ \cite{Biedermann:2016yds,Biedermann:2017bss}.
The contribution of order $\mathcal{O}{\left(\alphas\alpha^{6}\right)}$ is the second largest NLO contribution.
It is often referred as the QCD corrections to the VBS process.
In the following, this order is the one where our comparisons are focused on.
It is therefore usually referred to as simply \emph{NLO}.
As for the LO study, the various predictions are compared at the level of the cross section and differential distributions.

Finally, several predictions featuring parton shower are compared.
This allows to infer systematics differences between the various predictions.
This is the first time in the literature that such an analysis has been carried out \MP{True?}.

Obviously all VBS processes deserve such a study but the present article sets the standard for inferring systematics related to NLO corrections and beyond.


The article is organised as follow:
in the first part (section~\ref{sec:definition}), the process studied is defined.
Then, various approximation at LO and NLO are described in section~\ref{sec:details}.
It is followed by a presentation of the various programs used for the simulations.
Sections~\ref{sec:LO} and \ref{sec:NLO} are devoted to a LO and NLO study, respectively.
Section~\ref{sec:matching} deals with the matching to parton shower.
The last section (section \ref{sec:conclusion}) consists in concluding remarks and recommendations.
