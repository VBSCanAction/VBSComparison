Vector-boson scattering (VBS) is a class of processes that make it possible 
to probe the nature of Higgs-vector-vector couplings and triple and quartic gauge couplings.
It is usually understood that VBS refers to the scattering of massive vector bosons ($\PW^\pm,\PZ$), which therefore couple to the Higgs boson and can be longitudinally polarised.
The scattering of longitudinally-polarised bosons is of particular interest, because the corresponding matrix elements feature both gauge and unitarity cancellations that strongly depend on the actual structure of the Higgs sector.
A detailed study of this class of processes will therefore further constrain the Higgs couplings and hint at or exclude non-Standard Model Higgs bosons.

The process with two positively charged $\PW$ bosons is the VBS process with the largest signal-to-background ratio at the LHC, for which evidence has been found in $8 \TeV$ data~\cite{Aad:2014zda,Khachatryan:2014sta} and has been subsequently
observed~\cite{Sirunyan:2017ret} and measured~\cite{Aaboud:2016ffv} in $13\TeV$ data. At present 
the measurements of VBS processes are limited by statistics but the situation will change in a near future.
On the theoretical side, it is thus of prime importance to provide precise predictions and infer their related systematic errors.

The $\PW^+\PW^+$ scattering is the simplest VBS process to calculate, because of the double-charge structure of the leptonic final state that limits the number of partonic processes and total number of Feynman diagrams for each process.
It also implies that irreducible backgrounds are comparatively small, which make this VBS process experimentally favourable in comparison to \eg $\PW^+\PW^-$ scattering, which has the largest cross section.
Therefore, the $\PW^+\PW^+$ scattering is an ideal candidate for a detailed study of various theoretical predictions.

In the last few years, several next-to-leading order (NLO) computations became available for both the VBS process~\cite{Jager:2006zc,Jager:2006cp,Bozzi:2007ur,Jager:2009xx,Jager:2011ms,Denner:2012dz,Rauch:2016pai} and its QCD-induced irreducible background process~\cite{Rauch:2016pai,Melia:2010bm,Melia:2011gk,Campanario:2013gea,Baglio:2014uba}.
The VBS computations all rely on approximations, while recently the complete NLO corrections have been performed~\cite{Biedermann:2017bss}.
It is therefore interesting to infer in details the quality of the various approximations.
Indeed, apart from Ref.~\cite{Biedermann:2017bss} where it is commented on, \MP{more references?} no detailed comparison of the VBS approximations have been carried out.
Preliminary results of the present study have already been made public in Ref.~\cite{Anders:2018gfr}.

The hadronic process ito be considered is $\Pp\Pp\to\mu^+\nu_\mu{\rm e}^+\nu_{\rm e}\,\Pj\Pj+\mathrm{X}$, which includes the $\PW^+\PW^+$ scattering.
This final state possesses three contributions at leading order (LO) whose coupling orders are $\mathcal{O}{\left(\alpha^{6}\right)}$, $\mathcal{O}{\left(\alpha_{\rm s}\alpha^{5}\right)}$, and $\mathcal{O}{\left(\alphas^{2}\alpha^{4}\right)}$.
They are commonly referred to as electroweak (EW), interference, and QCD contributions, respectively.%
\footnote{The EW contribution is sometimes referred to as the VBS contribution even it possesses non-VBS contributions.}
Therefore, the present work starts with a LO study of these three contributions as a function of typical VBS cuts.
This allows to quantify the various contributions to the final state $\mu^+\nu_\mu{\rm e}^+\nu_{\rm e}\,\Pj\Pj$.
This is followed by a LO comparison between the various predictions at the level of the cross section and differential distributions.

At NLO, the process possesses four contributions of orders $\mathcal{O}{\left(\alpha^{7}\right)}$, $\mathcal{O}{\left(\alphas\alpha^{6}\right)}$, $\mathcal{O}{\left(\alphas^{2}\alpha^{5}\right)}$, and $\mathcal{O}{\left(\alphas^{3}\alpha^{4}\right)}$.
The largest ones are the EW corrections~\cite{Biedermann:2017bss,Biedermann:2016yds} of order $\mathcal{O}{\left(\alpha^{7}\right)}$.
The contribution to the order $\mathcal{O}{\left(\alphas\alpha^{6}\right)}$ is the second largest NLO contribution and is often referred to as the QCD corrections to the VBS process.
In the following, this order is the one where our comparisons are focused on and we will refer to it as simply \emph{NLO}.
As for the LO study, the various predictions are compared at the level of the cross section and differential distributions now at NLO accuracy.
In particular, this makes it possible to infer the accuracy of the so-called VBS approximation, which we will define in more details later.
To our knowledge, such a detailed study was still missing.

Finally, several predictions featuring parton shower are compared, giving the possibility to infer systematic differences between the various predictions.
A first study for the $\PW^+ \PW^- \Pj \Pj$ VBS process has been presented in
Ref.~\cite{Rauch:2016upa}, comparing the angular-ordered default shower and the dipole
shower and both{\sc MC@NLO}-like~\cite{Frixione:2002ik} and {\sc POWHEG}-like~\cite{Nason:2004rx,Frixione:2007vw} matching as implemented
in {\sc Herwig 7}~\cite{Bellm:2015jjp}, also showing scale-variation uncertainties. In this work we extend such a comparison by presenting 
results obtained by {\sc MadGraph5\_aMC@NLO}, {\sc Powheg} and {\sc Phantom} {\bf CITES}, possibly matched with different parton-showers.
This is the first time in the literature that NLO QCD calculations for VBS processed matched to parton shower are compared between different generators.

The article is organised as follows:
First, we define the process under study in Sec.~\ref{sec:definition}.
Various approximation at LO and NLO are described in Sec.~\ref{sec:details}.
This is followed by a presentation of the programs used for the computations.
Sections~\ref{sec:LO} and \ref{sec:NLO} are devoted to a LO and NLO study at fixed order, respectively.
Section~\ref{sec:matching} then adds the effect of parton showers and hadronization to the fixed-order results by matching the two.
The last section consists in concluding remarks and recommendations for experimental collaborations.
