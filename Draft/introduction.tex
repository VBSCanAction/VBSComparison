Vector-boson scattering (VBS) at a hadron collider 
usually refers to the interaction of massive vector-bosons ($\PW^\pm,\PZ$),
irradiated by partons (quarks) of the incoming protons, 
which in turn are deflected from the beam direction 
and enter the volume of the particle detectors.
As a consequence, the typical signature of VBS events
is characterised by two energetic jets 
and four fermions,
originating from the decay of the two vector bosons.
Among the possible diagrams,
the scattering process can be mediated also by a Higgs boson
and involve in particular their longitudinal component.
The interaction of of longitudinally polarised bosons is of particular interest, 
because the corresponding matrix elements feature both gauge and unitarity cancellations 
that strongly depend on the actual structure of the Higgs sector.
A detailed study of this class of processes will therefore further constrain the Higgs couplings 
at a very different energy scale with respect to the Higgs boson mass,
and hint at, or exclude, non-Standard Model behaviours.

The VBS process involving two same-sign $\PW$ bosons has the largest signal-to-background ratio at the LHC:
evidence for it was found at the centre-of-mass energy of $8~\TeV$ already~\cite{Aad:2014zda,Khachatryan:2014sta},
and it has been recently observed~\cite{Sirunyan:2017ret} and measured~\cite{Aaboud:2016ffv} 
at $13\TeV$ as well.
Presently, the measurements of VBS processes are limited by statistics, but the situation will change in a near future.
On the theoretical side, 
it is thus of prime importance to provide predictions with systematic uncertainties
at least comparable to the current and experimental precisions~\cite{CMS:2016rcn}.


The $\PW^+\PW^+$ scattering is also the simplest VBS process to calculate, 
because the double-charge structure of the leptonic final state 
limits the number of partonic processes and total number of Feynman diagrams for each process.
Therefore, this process is the ideal candidate for a comparative study of the different simulation tools.

In the last few years, several next-to-leading order (NLO) computations became available for both the VBS process~\cite{Jager:2006zc,Jager:2006cp,Bozzi:2007ur,Jager:2009xx,Jager:2011ms,Denner:2012dz,Rauch:2016pai} and its QCD-induced irreducible background process~\cite{Rauch:2016pai,Melia:2010bm,Melia:2011gk,Campanario:2013gea,Baglio:2014uba}.
The VBS computations all rely on approximations, while recently the complete NLO corrections have been performed~\cite{Biedermann:2017bss}.
It is therefore interesting to infer in details the quality of the various approximations.
Indeed, apart from Ref.~\cite{Biedermann:2017bss} where it is commented on, \MP{more references?} no detailed comparison of the VBS approximations have been carried out.
Preliminary results of the present study have already been made public in Ref.~\cite{Anders:2018gfr}.

The full gauge-invariant process including the $\PW^+\PW^+$ scattering 
 is $\Pp\Pp\to\mu^+\nu_\mu{\rm e}^+\nu_{\rm e}\,\Pj\Pj+\mathrm{X}$.
This final state receives three contributions at leading order (LO) whose coupling orders are $\mathcal{O}{\left(\alpha^{6}\right)}$, $\mathcal{O}{\left(\alpha_{\rm s}\alpha^{5}\right)}$, and $\mathcal{O}{\left(\alphas^{2}\alpha^{4}\right)}$.
They are commonly referred to as electroweak (EW), interference, and QCD contributions, respectively.%
\footnote{The EW contribution is sometimes referred to as the VBS contribution, even it involves also non-VBS contributions.}
Therefore, the present work starts with a LO study of these three contributions as a function of typical VBS cuts.
This allows to quantify the various contributions to the final state $\mu^+\nu_\mu{\rm e}^+\nu_{\rm e}\,\Pj\Pj$.
This is followed by a LO comparison between the various predictions at the level of the cross section and differential distributions.

At NLO, the process possesses four contributions of orders $\mathcal{O}{\left(\alpha^{7}\right)}$, $\mathcal{O}{\left(\alphas\alpha^{6}\right)}$, $\mathcal{O}{\left(\alphas^{2}\alpha^{5}\right)}$, and $\mathcal{O}{\left(\alphas^{3}\alpha^{4}\right)}$.
The largest ones are the EW corrections~\cite{Biedermann:2017bss,Biedermann:2016yds} of order $\mathcal{O}{\left(\alpha^{7}\right)}$.
The contribution to the order $\mathcal{O}{\left(\alphas\alpha^{6}\right)}$ is the second largest NLO contribution and is often referred to as the QCD corrections to the VBS process.
In the following, this order is the one where our comparisons are focused on and we will refer to it as simply \emph{NLO}.
As for the LO study, the various predictions are compared at the level of the cross section and differential distributions now at NLO accuracy.
In particular, this makes it possible to infer the accuracy of the so-called VBS approximation, which we will define in more details later.
To our knowledge, such a detailed study was still missing.

Finally, several predictions featuring parton shower are compared, giving the possibility to infer systematic differences between the various predictions.
A first study for the $\PW^+ \PW^- \Pj \Pj$ VBS process has been presented in
Ref.~\cite{Rauch:2016upa}, comparing the angular-ordered default shower and the dipole
shower and both{\sc MC@NLO}-like~\cite{Frixione:2002ik} and {\sc POWHEG}-like~\cite{Nason:2004rx,Frixione:2007vw} matching as implemented
in {\sc Herwig 7}~\cite{Bellm:2015jjp}, also showing scale-variation uncertainties. In this work we extend such a comparison by presenting 
results obtained by {\sc MadGraph5\_aMC@NLO}, {\sc Powheg} and {\sc Phantom} {\bf CITES}, possibly matched with different parton-showers.
This is the first time in the literature that NLO QCD calculations for VBS processed matched to parton shower are compared between different generators.

The article is organised as follows:
first, we define the process under study in Sec.~\ref{sec:definition}.
The various approximation which are provided by the different computer codes at LO and NLO are described in Sec.~\ref{sec:details}.
This is followed by a presentation of the programs used for the computations.
Sections~\ref{sec:LO} and \ref{sec:NLO} are devoted to a LO and NLO study at fixed order, respectively.
Section~\ref{sec:matching} complements our study by comparing predictions at LO and LO which include the effect of parton showers and hadronization.
The last section consists in concluding remarks and recommendations for experimental collaborations.
