Vector-boson scattering (VBS) at a hadron collider 
usually refers to the interaction of massive vector-bosons ($\PW^\pm,\PZ$),
radiated by partons (quarks) of the incoming protons, 
which in turn are deflected from the beam direction 
and enter the volume of the particle detectors.
As a consequence, the typical signature of VBS events
is characterised by two energetic jets 
and four fermions,
originating from the decay of the two vector bosons.
Among the possible diagrams,
the scattering process can be mediated by a Higgs boson
and involves in particular its longitudinal component.
The interaction of longitudinally polarised bosons is of particular interest, 
because the corresponding matrix elements feature unitarity cancellations 
that strongly depend on the actual structure of the Higgs sector of the Standard Model (SM).
A detailed study of this class of processes will therefore further constrain the Higgs couplings 
at a very different energy scale with respect to the Higgs boson mass,
and hint at, or exclude, non-Standard Model behaviours.

The VBS process involving two same-sign $\PW$ bosons has the largest signal-to-background ratio of all the VBS processes at the Large Hadron Collider (LHC):
evidence for it was found at the centre-of-mass energy of $8~\TeV$ already~\cite{Aad:2014zda,Khachatryan:2014sta},
and it has been recently observed~\cite{Sirunyan:2017ret} and measured~\cite{Aaboud:2016ffv} 
at $13\TeV$ as well.
Presently, the measurements of VBS processes are limited by statistics, but the situation will change in the near future.
On the theoretical side, 
it is thus of prime importance to provide predictions with systematic uncertainties
at least comparable to the current and envisaged experimental precision~\cite{CMS:2016rcn}.


$\PW^+\PW^+$ scattering is also the simplest VBS process to calculate, 
because the double-charge structure of the leptonic final state 
limits the number of partonic processes and total number of Feynman diagrams for each process.
Nonetheless, it possesses all features of VBS at the LHC and is thus representative of other VBS signatures.
Therefore, it is the ideal candidate for a comparative study of the different simulation tools.

In the last few years, several next-to-leading order (NLO) computations have become available for both the VBS process~\cite{Jager:2006zc,Jager:2006cp,Bozzi:2007ur,Jager:2009xx,Jager:2011ms,Denner:2012dz,Rauch:2016pai} and its QCD-induced irreducible background process~\cite{Rauch:2016pai,Melia:2010bm,Melia:2011gk,Campanario:2013gea,Baglio:2014uba}.
All these VBS computations rely on various approximations, typically neglecting contributions which are expected to be small under realistic experimental setups~\cite{Denner:2012dz,Oleari:2003tc}.
Recently, the complete NLO corrections to $\PW^+\PW^+$ have been evaluated in Ref.~\cite{Biedermann:2017bss}, 
making it possible for the first time to study in detail the quality of the VBS approximations at NLO QCD.\footnote{Preliminary results of the present study have already been made public in Ref.~\cite{Anders:2018gfr}. 
A similar study has also been made public very recently for the EW production of a Higgs boson in association with 3 jets \cite{Campanario:2018ppz}.}

% {\bf MZ: we can remove the following sentence, up to the citation of Oleari}
% Apart from Ref.~\cite{Biedermann:2017bss}, no detailed comparison of the VBS approximations have been carried out beyond the leading-order~\cite{Oleari:2003tc}. 

\iffalse
The full gauge-invariant process including the $\PW^+\PW^+$ scattering 
 is $\Pp\Pp\to\mu^+\nu_\mu{\rm e}^+\nu_{\rm e}\,\Pj\Pj+\mathrm{X}$.
This final state receives three contributions at leading order (LO) whose coupling orders are $\mathcal{O}{\left(\alpha^{6}\right)}$, $\mathcal{O}{\left(\alpha_{\rm s}\alpha^{5}\right)}$, and $\mathcal{O}{\left(\alphas^{2}\alpha^{4}\right)}$.
They are commonly referred to as electroweak (EW), interference, and QCD contributions, respectively.%
\footnote{The EW contribution is sometimes referred to as the VBS contribution, even it involves also non-VBS contributions.}
\fi
This article starts with the definition of the VBS process in Sec.~\ref{sec:definition}.
After having described the approximations of the various computer 
codes in Sec.~\ref{sec:details}, in Sec.~\ref{sec:LO} a leading-order (LO) study 
of the different contributions which lead to the production of two same-sign $\PW$ bosons and 
two jets is performed, as a function of typical VBS cuts. In the same section predictions for VBS from different tools are compared at 
the level of the cross section and differential distributions. The article continues in Sec.~\ref{sec:NLO} where the comparison is extended to the
NLO corrections to VBS. The effect of the inclusion of matching LO and NLO computations to parton shower (PS) is 
discussed in Sec.~\ref{sec:matching}. Finally,
Sec.~\ref{sec:conclusion} contains a summary of the article and concluding remarks.


\iffalse
At NLO, the process possesses four contributions of orders $\mathcal{O}{\left(\alpha^{7}\right)}$, $\mathcal{O}{\left(\alphas\alpha^{6}\right)}$, $\mathcal{O}{\left(\alphas^{2}\alpha^{5}\right)}$, and $\mathcal{O}{\left(\alphas^{3}\alpha^{4}\right)}$.
The largest ones are the EW corrections~\cite{Biedermann:2017bss,Biedermann:2016yds} of order $\mathcal{O}{\left(\alpha^{7}\right)}$.
The contribution to the order $\mathcal{O}{\left(\alphas\alpha^{6}\right)}$ is the second largest NLO contribution and is often referred to as the QCD corrections to the VBS process.
The main focus of the paper is the comparison of these corrections, and we will henceforth refer to this order simply as \emph{NLO}.
As for the LO study, the various predictions are compared at the level of the cross section and differential distributions now at NLO accuracy.
In particular, this makes it possible to infer the accuracy of the so-called VBS approximation, which we will define in more details later.
To our knowledge, such a detailed study was still missing. \AK{Vague statement. It is either missing or not. I suggest simply removing the sentence.}

Finally, several parton shower matched predictions are compared, giving the possibility to infer systematic differences between the various parton showers and matching prescriptions.
A similar study for the $\PW^+ \PW^- \Pj \Pj$ VBS process has been presented in
Ref.~\cite{Rauch:2016upa}. Here a comparison of the default angular-ordered shower and the dipole
shower  in {\sc Herwig 7}~\cite{Bellm:2015jjp} was carried out. Both {\sc MC@NLO}-like~\cite{Frixione:2002ik} and {\sc Powheg}-like~\cite{Nason:2004rx,Frixione:2007vw} matching was studied. In this work we extend such a comparison by presenting 
results obtained by {\sc MadGraph5\_aMC@NLO}, {\sc Powheg} and {\sc Phantom} {\bf CITES} for the $\PW^+ \PW^+ \Pj \Pj$ VBS process, possibly\AK{Why possibly?} matched with different parton-showers.
This is the first time in the literature that NLO QCD calculations for VBS processes matched to parton shower are compared between different generators. 

The article is organised as follows:
In Sec.~\ref{sec:definition}, we define the VBS process.
The various approximation which are provided by the different computer codes at LO and NLO are described in Sec.~\ref{sec:details}.
This is followed by a presentation of the programs used for the computations.
Sections~\ref{sec:LO} and \ref{sec:NLO} are devoted to a LO and NLO study at fixed order, respectively.
Section~\ref{sec:matching} complements our study by comparing predictions at LO and NLO which include the effect of parton showers and hadronisation.
The last section consists of concluding remarks and recommendations for experimental collaborations.
\fi
