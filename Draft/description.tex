\begin{table*}[ht!]
    \footnotesize
    \begin{tabularx}{\textwidth}{c|c|X|X|X|X|X}
        Code  &  $\mathcal O(\alpha^6)$ $|s|^2/$ $|t|^2/|u|^2$  &  $\mathcal O(\alpha^6)$ interf.  &  Non-res.  & NLO &  NF QCD  &  EW corr. to $\mathcal O(\alpha^5\alpha_s)$  \\
        \hline
        \hline
        {\sc Bonsay}        &  $t/u$    &  No       &  Yes, virt. No    &  Yes   & No       &  No  \\
        {\sc POWHEG}        &  $t/u$    &  No       &  Yes              &  Yes   & No       &  No  \\
        {\sc MG5\_aMC}      &  Yes      &  Yes      &  Yes              &  Yes   & No virt. &  No \\
        {\sc MoCaNLO+Recola}&  Yes      &  Yes      &  Yes              &  Yes   & Yes      &  Yes  \\
        {\sc PHANTOM}       &  Yes      &  Yes      &  Yes              &  No    & -        & - \\  
        {\sc VBFNLO}        &  Yes      &  No       &  Yes              &  Yes   & No       &  No  \\
        {\sc Whizard}       &  Yes      &  Yes      &  Yes              &  No    & -        & - \\  
    \end{tabularx}
    \caption{\label{tab:wg1_codes} Summary of the different properties of the codes employed in the comparison.}
\end{table*}

In the comparison, the following codes are used: 
\begin{itemize}
 \item The program {\sc Bonsay} consists of a general-purpose Monte Carlo integrator
and matrix elements taken from several sources: Born matrix elements are
adapted from the program {\sc Lusifer} \cite{Dittmaier:2002ap} for the partonic
processes, real matrix elements are written by Marina Billoni, and virtual
matrix elements by Stefan Dittmaier.
One loop integrals are evaluated using the {\sc Collier} library
\cite{Denner:2014gla,Denner:2016kdg}.

  \item {\sc MadGraph5\_aMC@NLO}~\cite{Alwall:2014hca} is an automatic meta-code (a code that generates codes) which makes it possible to simulate any scattering process
      including NLO QCD corrections both at fixed order and including matching to parton showers. It makes use of the FKS subtraction method~\cite{Frixione:1995ms,
        Frixione:1997np} (automated in the module {\sc MadFKS}~\cite{Frederix:2009yq,
        Frederix:2016rdc}) for regulating IR singularities. The computationsof one-loop amplitudes are carried out by switching dynamically between 
        two integral-reduction techniques, OPP~\cite{Ossola:2006us} or Laurent-series expansion~\cite{Mastrolia:2012bu},
        and TIR~\cite{Passarino:1978jh,Davydychev:1991va,Denner:2005nn}. These have been automated in the module {\sc MadLoop}~\cite{Hirschi:2011pa}, which 
        in turn exploits {\sc CutTools}~\cite{Ossola:2007ax}, {\sc Ninja}~\cite{Peraro:2014cba,
        Hirschi:2016mdz}, or {\sc IREGI}~\cite{ShaoIREGI}, together with an in-house implementation of the {\sc OpenLoops} optimisation~\cite{Cascioli:2011va}.\\
        The simulation of VBS at NLO-QCD accuracy can be performed by issuing the following commands in the program interface:
\begin{verbatim}
> set complex_mass_scheme #1
> import model loop_qcd_qed_sm_Gmu #2
> generate p p > e+ ve mu+ vm j j QCD=0 [QCD] #3
> output #4
\end{verbatim}
  With these commands the complex-mass scheme is turned on {\tt \#1}, then the NLO-capable model is loaded {\tt \#2}\footnote{Despite
            the {\tt loop\_qcd\_qed\_sm\_Gmu} model also includes NLO counterterms for computing electro-weak corrections, it is not yet possible to compute such corrections 
        with the current version of the code.}, finally the process code is generated {\tt \#3} (note the {\tt QCD=0} syntax to select the purely-electroweak process)
        and written to disk {\tt \#4}. Because of some internal limitations, which will be lifted in the future version capable of computing both QCD and EW corrections, 
        only loops with QCD-interacting particles are generated.
        \MP{Detail of the approximation done, divergent part, assumed to cancel etc.}
        
  \item {{\sc{Phantom}}~\cite{Ballestrero:2007xq} is a dedicated tree--level Monte Carlo for six parton final states at $pp,\, p\bar{p}$
  and $e^+e^-$ colliders at $\alpha_{ew}^6$ and $\alpha_{ew}^4\alpha_s^2$ including interferences between the two sets of diagrams.
It employs complete tree--level matrix elements in the complex--mass scheme~\cite{Denner:2006ic} computed via the modular helicity 
formalism~\cite{Ballestrero:1999md,Ballestrero:1994jn}. The integration uses a multichannel approach~\cite{BERENDS1985441} and an adaptive
strategy~\cite{PETERLEPAGE1978192}. {\sc{Phantom}} generates unweighted events at parton--level for both the SM and a few instances of BSM theories.}      

  \item The {\sc Powheg-Box}~\cite{Alioli:2010xd,Frixione:2007vw} is a framework for mathcing NLO-QCD calculations with parton showers.
It relies on the user providing the matrix elements and Born phase space, but will automaticaly construct FKS \cite{Frixione:1995ms} subtraction terms and the phase space for the real emission.
For the VBS processes all matrix elements are being provided by a previous version of {\sc VBFNLO}~\cite{Arnold:2008rz, Arnold:2011wj, Baglio:2014uba} and hence the approximations used in the {\sc Powheg-Box} are the similar to those used in {\sc VBFNLO}.
\MP{Mention the non-clustering for the scale as well as the different running of alphas at NLO.}

  \item The program {\sc MoCaNLO+Recola} is made of a flexible Monte Carlo program dubbed {\sc MoCaNLO}~\cite{MoCaNLO} and the general matrix element generator {\sc Recola} \cite{Actis:2012qn,Actis:2016mpe}.
To numerically evaluate the one-loop scalar and tensor integrals, {\sc Recola} relies on the {\sc Collier} library \cite{Denner:2014gla,Denner:2016kdg},
These tools have been successfully used for the computation of NLO corrections for VBS~\cite{Biedermann:2016yds,Biedermann:2017bss}.

  \item {\sc VBFNLO}~\cite{Arnold:2008rz, Arnold:2011wj, Baglio:2014uba} is a flexible
parton-level Monte Carlo for processes with electroweak bosons. It
allows the calculation of VBS processes at NLO QCD in the VBF
approximation and including the s-channel triboson contribution,
neglecting interferences between the two. Besides the SM, also anomalous
couplings of the Higgs and gauge bosons can be simulated.

  \item {\sc Whizard}~\cite{Moretti:2001zz,Kilian:2007gr} is a multi-purpose
event generator with the LO matrix element generator {\sc O'Mega}. It
provides FKS subtraction terms for any NLO process, while virtual matrix
elements are provided externally by {\sc
OpenLoops}~\cite{Cascioli:2011va} (alternatively, {\sc Recola}~\cite{Actis:2012qn,Actis:2016mpe}
(cf. above) can be used as well). {\sc Whizard} allows to simulate a
huge number of BSM models as well, in particular for new physics in
the VBS channel in terms of both higher-dimensional operators as well as explicit
resonances.

\end{itemize}

% The complete comparison of the codes will be published in a separate work. Here, we present some preliminary results obtained at LO ($\mathcal O (\alpha^6)$) and including
NLO QCD corrections at fixed-order $\mathcal O (\alpha^6\alpha_s)$, for the process ${\rm p}{\rm p}\to\mu^+\nu_\mu{\rm e}^+\nu_{\rm e}{\rm j}{\rm j}$.
In Tab.~\ref{tab:wg1_codes} the details of the various codes are reported. In particular, it is specified whether
\begin{itemize}
    \item all $s$- and $t/u$-channel diagrams that lead to the considered final state are included;
    \item interferences between diagrams are included at LO;
    \item diagrams which do not feature two resonant vector bosons are included;
    \item the so-called non-factorisable (NF) QCD corrections, that is the corrections where (real or virtual) gluons are exchanged between different quark lines,
        are included;
    \item EW corrections to the $\mathcal O (\alpha^5\alpha_s)$ interference are included. These corrections are of the same order as the NLO QCD corrections to
        the  $\mathcal O (\alpha^6$) term.
\end{itemize}
%
%
