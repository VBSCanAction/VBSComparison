In the following, the codes employed throughout this paper and the approximations implemented in each of them will be discussed:

\begin{itemize}

\item The program {\sc Bonsay} consists of a general-purpose Monte Carlo integrator written by Christopher Schwan and matrix elements taken from different sources:
Born matrix elements are adapted from the program {\sc Lusifer}~\cite{Dittmaier:2002ap}, real matrix elements are written by Marina Billoni, and virtual matrix elements by Stefan Dittmaier.
One loop integrals are evaluated using the {\sc Collier} library~\cite{Denner:2014gla,Denner:2016kdg}.
For the fiducial cross sections it uses the VBS approximation at LO and NLO.
The virtual corrections are additionally approximated using a double-pole approximation.
For more inclusive cross sections at LO the exact matrix elements ($s$-channels, interferences) can also be used.

  \item {\sc MadGraph5\_aMC@NLO}~\cite{Alwall:2014hca} (henceforth {\sc MG5\_aMC}) is an automatic meta-code (a code that generates codes) which makes it possible to simulate any scattering process
      including NLO QCD corrections both at fixed order and including matching to parton showers, using the {\sc MC@NLO}\ method~\cite{Frixione:2002ik}. It makes use of the subtraction method by Frixione, Kunszt and Signer (FKS)~\cite{Frixione:1995ms,
        Frixione:1997np} (automated in the module {\sc MadFKS}~\cite{Frederix:2009yq,
        Frederix:2016rdc}) for regulating IR singularities. The computations of one-loop amplitudes are carried out by switching dynamically between
        two integral-reduction techniques, OPP~\cite{Ossola:2006us} or Laurent-series expansion~\cite{Mastrolia:2012bu},
        and tensor-integral reduction~\cite{Passarino:1978jh,Davydychev:1991va,Denner:2005nn}. These have been automated in the module {\sc MadLoop}~\cite{Hirschi:2011pa}, which
        in turn exploits {\sc CutTools}~\cite{Ossola:2007ax}, {\sc Ninja}~\cite{Peraro:2014cba,
        Hirschi:2016mdz}, {\sc IREGI}~\cite{ShaoIREGI}, or {\sc Collier}~\cite{Denner:2016kdg}, together with an in-house 
        implementation of the {\sc OpenLoops} optimisation~\cite{Cascioli:2011va}. Finally, scale and PDF uncertainties can be obtained in an exact manner via reweighting
        at negligible additional CPU cost~\cite{Frederix:2011ss}.\\
        The simulation of VBS at NLO-QCD accuracy can be performed by issuing the following commands in the program interface:
\begin{verbatim}
> set complex_mass_scheme
> import model loop_qcd_qed_sm_Gmu
> generate p p > e+ ve mu+ vm j j QCD=0 [QCD]
> output
\end{verbatim}
  With these commands the complex-mass scheme is turned on, then the NLO-capable model is loaded\footnote{Despite
            the {\tt loop\_qcd\_qed\_sm\_Gmu} model also includes NLO counterterms for computing EW corrections, it is not yet possible to compute such corrections
        with the current public version of the code.}, finally the process code is generated (note the {\tt QCD=0} syntax to select the purely-EW process)
        and written to disk. No approximation is performed for the Born and real-emission matrix elements. 
        For what concerns the virtual matrix element, because of some internal limitations which will be lifted in the future version capable of computing both QCD and EW corrections,
        only loops with QCD-interacting particles are generated. Such an approximation is equivalent to the assumption that the finite part of
        those loops which feature EW bosons is zero. In practice, since a part of the contribution to the single pole is also missing, the internal 
        pole-cancellation check at run time has to be turned off, by setting the value of the {\tt IR\-Pole\-Check\-Threshold} and 
        {\tt Precision\-Virtual\-At\-Run\-Time} parameters in the {\tt Cards\-/FKS\_\-params.dat} file to -1.

\item The program {\sc MoCaNLO+Recola} is made of a flexible Monte Carlo program dubbed {\sc MoCaNLO} and of the matrix element generator {\sc Recola}~\cite{Actis:2012qn,Actis:2016mpe}.
It can compute arbitrary processes for the LHC at both NLO QCD and EW accuracy in the Standard Model.
This is made possible by the fact that {\sc Recola} can compute arbitrary processes at tree and one-loop level in the Standard Model.
To that end, it relies on the {\sc Collier} library \cite{Denner:2014gla,Denner:2016kdg} to numerically evaluate the one-loop scalar and tensor integrals.
In addition, the subtraction of the IR divergences appearing in the real corrections has been automatised thanks to the Catani--Seymour dipole formalism for both QCD and QED \cite{Catani:1996vz,Dittmaier:1999mb}.
The code has demonstrated its ability to compute at NLO high multiplicity processes up to $2 \to 7$ \cite{Denner:2015yca,Denner:2016wet}.
In particular the full NLO corrections to VBS and its irreducible background \cite{Biedermann:2016yds,Biedermann:2017bss} have been obtained thanks to this tool.
One key aspect for these high multiplicity processes is the fast integration which is ensured by using similar phase-space mappings to those of Refs.~\cite{Berends:1994pv,Denner:1999gp,Dittmaier:2002ap}. 
In {\sc MoCaNLO+Recola} no approximation is performed neither at LO nor at NLO.
It implies that, also contributions stemming from EW corrections to the interference are computed.
        
  \item {\sc Phantom}~\cite{Ballestrero:2007xq} is a dedicated tree-level Monte Carlo for six parton final states 
  at $\Pp \Pp,\, \Pp\bar{\Pp}$ and $\Pe^+\Pe^-$ colliders at orders $\mathcal O(\alpha^6)$ and $\mathcal O(\alphas^2\alpha^4)$ including interferences between the two sets of diagrams.
It employs complete tree-level matrix elements in the complex-mass scheme~\cite{Denner:1999gp,Denner:2005fg,Denner:2006ic} computed via the modular helicity formalism~\cite{Ballestrero:1999md,Ballestrero:1994jn}.
The integration uses a multichannel approach~\cite{Berends:1984gf} and an adaptive strategy~\cite{Lepage:1977sw}.
{\sc Phantom} generates unweighted events at parton level for both the SM and a few instances of beyond the Standard Model (BSM) theories.

  \item The {\sc Powheg-Box}~\cite{Nason:2004rx,Frixione:2007vw,Alioli:2010xd} is a framework for matching NLO-QCD calculations with parton showers.
It relies on the user providing the matrix elements and Born phase-space, but will automatically construct FKS \cite{Frixione:1995ms} subtraction terms and the phase space for the real emission.
For the VBS processes all matrix elements are being provided by a previous version of {\sc VBFNLO}~\cite{Arnold:2008rz, Arnold:2011wj, Baglio:2014uba} and hence the approximations used in the {\sc Powheg-Box} are similar to those used in {\sc VBFNLO}.

  \item {\sc VBFNLO}~\cite{Arnold:2008rz, Arnold:2011wj, Baglio:2014uba} is a flexible
    parton-level Monte Carlo for processes with EW bosons. It
    allows the calculation of VBS processes at NLO QCD in the VBS
    approximation, with process IDs between 200 and 290. The corresponding
    $s$-channel contributions are available separately as tri-boson processes with
    semi-leptonic decays, with process IDs in the 400 range\ZM{which range is it?}. These can simply
    be added on top of the VBS contribution. Interferences between the two are therefore neglected.
    The usage of leptonic tensors in the calculation, pioneered in
    Ref.~\cite{Jager:2006zc}, thereby leads to a significant speed improvement over
    automatically generated code.  Besides the SM, also a variety of
    new-physics models including anomalous couplings of the Higgs and gauge
    bosons can be simulated.

  \item {\sc Whizard}~\cite{Moretti:2001zz,Kilian:2007gr} is a multi-purpose
      event generator with the LO matrix element generator {\sc O'Mega}. \ZM{ if NLO results for this processes cannot be provided, we should skip what follows, or at least clarify the limitations}
provides FKS subtraction terms for any NLO process, while virtual matrix
elements are provided externally by {\sc
OpenLoops}~\cite{Cascioli:2011va} (alternatively, {\sc Recola}~\cite{Actis:2012qn,Actis:2016mpe}
(\emph{cf.}\ above) can be used as well). {\sc Whizard} allows to simulate a
huge number of BSM models as well, in particular in
the VBS channel in terms of both higher-dimensional operators as well as explicit
resonances.

\end{itemize}

We conclude this section by summarising the characteristics of the various codes in Tab.~\ref{tab:wg1_codes}.
In particular, it is specified whether
\begin{itemize}
    \item all $s$- and $t/u$-channel diagrams are included;
    \item interferences between different diagrams of diagrams ($s/t/u$-channel) are included at LO;
    \item diagrams which do not feature two resonant vector-bosons are included;
    \item the so-called non-factorisable (NF) QCD corrections, \emph{i.e.}\ the corrections where (real or virtual) gluons are exchanged between different quark lines,
        are included;
    \item EW corrections to the interference of order $\mathcal O (\alpha^5\alphas)$ are included.
    These corrections are of the same order as the NLO QCD corrections to the contribution of order $\mathcal O (\alpha^6$) term.
\end{itemize}

\begin{table*}[ht!]
    \footnotesize
    \begin{tabularx}{\textwidth}{c|c|X|X|X|X|X}
        Code  &  $\mathcal O(\alpha^6)$ $s, t, u$  &  $\mathcal O(\alpha^6)$ interf.  &  Non-res.  & NLO &  NF QCD  &  EW corr. to order $\mathcal O(\alphas \alpha^5)$  \\
        \hline
        \hline
        {\sc Bonsay}        &  $t/u$    &  No       &  Yes, virt. No    &  Yes   & No       &  No  \\
        {\sc Powheg}        &  $t/u$    &  No       &  Yes              &  Yes   & No       &  No  \\
        {\sc MG5\_aMC}      &  Yes      &  Yes      &  Yes              &  Yes   & virt. No &  No \\
        {\sc MoCaNLO+Recola}&  Yes      &  Yes      &  Yes              &  Yes   & Yes      &  Yes  \\
        {\sc PHANTOM}       &  Yes      &  Yes      &  Yes              &  No    & -        & - \\
        {\sc VBFNLO}        &  Yes      &  No       &  Yes              &  Yes   & No       &  No  \\
        {\sc Whizard}       &  Yes      &  Yes      &  Yes              &  No    & -        & - \\
    \end{tabularx}
    \caption{\label{tab:wg1_codes} Summary of the different properties of the computer programs employed in the comparison.}
\end{table*}
