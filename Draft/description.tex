In the comparison, the following codes are used:

\begin{itemize}

 \item The program {\sc Bonsay} consists of a general-purpose Monte Carlo integrator and matrix elements taken from several sources.
 Born matrix elements are adapted from the program {\sc Lusifer} \cite{Dittmaier:2002ap} for the partonic processes, real matrix elements are written by Marina Billoni, and virtual matrix elements by Stefan Dittmaier.
One loop integrals are evaluated using the {\sc Collier} library~\cite{Denner:2014gla,Denner:2016kdg}.

  \item {\sc MadGraph5\_aMC@NLO}~\cite{Alwall:2014hca} is an automatic meta-code (a code that generates codes) which makes it possible to simulate any scattering process
      including NLO QCD corrections both at fixed order and including matching to parton showers. It makes use of the FKS subtraction method~\cite{Frixione:1995ms,
        Frixione:1997np} (automated in the module {\sc MadFKS}~\cite{Frederix:2009yq,
        Frederix:2016rdc}) for regulating IR singularities. The computations of one-loop amplitudes are carried out by switching dynamically between
        two integral-reduction techniques, OPP~\cite{Ossola:2006us} or Laurent-series expansion~\cite{Mastrolia:2012bu},
        and TIR~\cite{Passarino:1978jh,Davydychev:1991va,Denner:2005nn}. These have been automated in the module {\sc MadLoop}~\cite{Hirschi:2011pa}, which
        in turn exploits {\sc CutTools}~\cite{Ossola:2007ax}, {\sc Ninja}~\cite{Peraro:2014cba,
        Hirschi:2016mdz}, {\sc IREGI}~\cite{ShaoIREGI}, or {\sc Collier}~\cite{Denner:2016kdg}, together with an in-house 
        implementation of the {\sc OpenLoops} optimisation~\cite{Cascioli:2011va}. Finally, scale and PDF uncertainties can be obtained in an exact manner via reweigthing
        at zero additional CPU cost~\cite{Frederix:2011ss}.\\
        The simulation of VBS at NLO-QCD accuracy can be performed by issuing the following commands in the program interface:
\begin{verbatim}
> set complex_mass_scheme #1
> import model loop_qcd_qed_sm_Gmu #2
> generate p p > e+ ve mu+ vm j j QCD=0 [QCD] #3
> output #4
\end{verbatim}
  With these commands the complex-mass scheme is turned on {\tt \#1}, then the NLO-capable model is loaded {\tt \#2}\footnote{Despite
            the {\tt loop\_qcd\_qed\_sm\_Gmu} model also includes NLO counterterms for computing electro-weak corrections, it is not yet possible to compute such corrections
        with the current version of the code.}, finally the process code is generated {\tt \#3} (note the {\tt QCD=0} syntax to select the purely-electroweak process)
        and written to disk {\tt \#4}. Because of some internal limitations, which will be lifted in the future version capable of computing both QCD and EW corrections,
        only loops with QCD-interacting particles are generated. As it has been already mentioned, such an approximation is equivalent to the assumption that the finite part of
        those loops which feature EW bosons is zero. In practice, since a part of the contribution to the single pole is also missing, the internal 
        pole-cancelation check at run time has to be turned off, by setting the value of the {\tt IR\-Pole\-Check\-Threshold} and 
        {\tt Precision\-Virtual\-At\-Run\-Time} parameters in the {\tt Cards/FKS\_params.dat} file to -1.

\item The program {\sc MoCaNLO+Recola} is made of a flexible Monte Carlo program dubbed {\sc MoCaNLO} and of the matrix element generator {\sc Recola}~\cite{Actis:2012qn,Actis:2016mpe}.
It can compute arbitrary processes for the LHC at both NLO QCD and EW accuracy in the Standard Model.
This is made possible by the fact that {\sc Recola} can compute arbitrary processes at tree and one-loop level in the Standard Model.
To that end, it relies on the {\sc Collier} library \cite{Denner:2014gla,Denner:2016kdg} to numerically evaluate the one-loop scalar and tensor integrals.
In addition, the subtraction of the IR divergences appearing in the real corrections has been automatised thanks to the Catani--Seymour dipole formalism for both QCD and QED \cite{Catani:1996vz,Dittmaier:1999mb}.
The code has demonstrated its ability to compute at NLO high multiplicity processes up to $2 \to 7$ \cite{Denner:2015yca,Denner:2016wet}.
In particular the full NLO corrections to VBS and its irreducible background \cite{Biedermann:2016yds,Biedermann:2017bss} have been obtained from this tool.
One key aspect for these high multiplicity processes is the fast integration which is ensured by using similar phase-space mappings to those of Refs.~\cite{Berends:1994pv,Denner:1999gp,Dittmaier:2002ap}.
        
  \item {\sc Phantom}~\cite{Ballestrero:2007xq} is a dedicated tree-level Monte Carlo for six parton final states 
  at $\Pp \Pp,\, \Pp\bar{\Pp}$ and $\Pe^+\Pe^-$ colliders at orders $\alpha^6$ and $\alphas^2\alpha^4$ including interferences between the two sets of diagrams.
It employs complete tree-level matrix elements in the complex-mass scheme~\cite{Denner:2006ic} computed via the modular helicity formalism~\cite{Ballestrero:1999md,Ballestrero:1994jn}.
The integration uses a multichannel approach~\cite{Berends:1984gf} and an adaptive strategy~\cite{Lepage:1977sw}.
{\sc Phantom} generates unweighted events at parton level for both the SM and a few instances of BSM theories.

  \item The {\sc Powheg-Box}~\cite{Alioli:2010xd,Frixione:2007vw,Nason:2006hfa} is a framework for matching NLO-QCD calculations with parton showers.
It relies on the user providing the matrix elements and Born phase space, but will automatically construct FKS \cite{Frixione:1995ms} subtraction terms and the phase space for the real emission.
For the VBS processes all matrix elements are being provided by a previous version of {\sc VBFNLO}~\cite{Arnold:2008rz, Arnold:2011wj, Baglio:2014uba} and hence the approximations used in the {\sc Powheg-Box} are the similar to those used in {\sc VBFNLO}.
The {\sc Powheg-Box} uses its own implementation of the two loop running for $\alpha_{\rm s}$.
The renormalisation and factorisation scale used differ slightly from one defined in Eq.~\eqref{eq:defscale}, as rather than constructing the jets the {\sc Powheg-Box} uses the transverse momentum of the two final-state quarks in the underlying Born event. 

  \item {\sc VBFNLO}~\cite{Arnold:2008rz, Arnold:2011wj, Baglio:2014uba} is a flexible
    parton-level Monte Carlo for processes with electroweak bosons. It
    allows the calculation of VBS processes at NLO QCD in the VBS
    approximation, with process IDs between 200 and 290. The corresponding
    s-channel contributions are available separately as triboson processes with
    semi-leptonic decays, with process IDs in the 400 range. These can simply
    be added on top of the VBS contribution, as interferences between the two are neglected.
    The usage of leptonic tensors in the calculation, pioneered in
    Ref.~\cite{Jager:2006zc}, thereby leads to a significant speed improvement over
    automatically generated code.  Besides the SM, also a variety of
    new-physics models including anomalous couplings of the Higgs and gauge
    bosons can be simulated.

  \item {\sc Whizard}~\cite{Moretti:2001zz,Kilian:2007gr} is a multi-purpose
event generator with the LO matrix element generator {\sc O'Mega}. It
provides FKS subtraction terms for any NLO process, while virtual matrix
elements are provided externally by {\sc
OpenLoops}~\cite{Cascioli:2011va} (alternatively, {\sc Recola}~\cite{Actis:2012qn,Actis:2016mpe}
(cf. above) can be used as well). {\sc Whizard} allows to simulate a
huge number of BSM models as well, in particular for new physics in
the VBS channel in terms of both higher-dimensional operators as well as explicit
resonances.

\end{itemize}
