\begin{abstract}
  \noindent

  Vector-boson scattering processes are of great importance for the
  current run-2 and future runs at design energy of the Large Hadron
  Collider. The presence of triple and quartic gauge couplings in the process
  gives access the gauge sector of the Standard Model and possible new-physics
  contributions there. 
  To test this hypothesis, a precise knowledge of the Standard
  Model contributions is necessary, 
  with a precision which at least matches the
  experimental uncertainties
  of existing and forthcoming measurements.
  In this article, we present a detailed study of the vector-boson
  scattering process with two positively-charged leptons and missing
  transverse momentum in the final state, mediated predominantly by
  same-sign production of two W bosons with positive charge.
  In particular, we first carry out a systematic comparison of the various 
  approximations that are usually performed for this kind of process against the complete calculation, 
  at LO and NLO QCD accuracy. Such  a study is performed both 
  in the usual fiducial region used by experimental collaborations and in a more inclusive phase space,
  where the differences among the various approximations lead to more sizeable effects.
  Afterwards, we turn ourselves to predictions matched to parton showers, at LO and NLO: we show that on the
  one hand, the inclusion of NLO QCD corrections leads to more stable predictions, but on the other the details 
  of the matching and of the parton-shower programs lead to differences which are considerably larger than those
  observed at fixed-order, even in the experimental fiducial region. 
  We conclude with some recommendations for experimental studies of vector-boson scattering processes.

\end{abstract}
\thispagestyle{empty}
\vfill
\newpage
\setcounter{page}{1}

%\tableofcontents
\newpage
