As mentioned previously, the contribution of main interest in our process is
the scattering of two W gauge bosons, which includes the quartic gauge-boson vertex.
Therefore it is justified to approximate the full EW contributions simply by only these contributions which contain the $2\rightarrow 2$ scattering process as a subpart.
However, this set of contributions is not gauge invariant.
To make it gauge invariant, one should perform an on-shell projection of the incoming W bosons.
Unfortunately, these momenta are space-like and thus a simple on-shell projection is not possible.
Instead, one can keep the W boson legs connected to the external quark line off-shell while the ones connected to the final-state leptons, which are already time-like, are put on-shell.
Then the polarisation of the gauge boson is accommodated following the implementations of Refs.~\cite{Kuss:1995yv,Accomando:2006hq}.
Such an approximation is usually called effective vector-boson approximation (EVBA) \cite{Dawson:1984gx,Duncan:1985vj,Cahn:1983ip}.

A less crude approximation consists in considering all $t$- and $u$- diagrams and squaring them separately, neglecting interference contributions between the two.
These interferences are expected to be small in the VBS fiducial region, as they are both phase-space and colour suppressed.
The $s$-channel squared diagrams and any interferences with $s$-channels are left out as well.
This approximation is often called $t$-/$u$- approximation, VBF, or even VBS approximation.
We will adopt the latter denomination in the following of the article.
Such an approximation is implemented at LO in the computer codes {\sc Bonsay} and {\sc Powheg}.
This approximation is gauge-invariant, which can be seen by considering that the two protons and therefore the two incoming quarks belong to two different, but otherwise identical, copies of the $SU \left(3\right)$ gauge group.

The squared matrix element of the $s$-channel contributions can be added in addition, but all interferences between different kinematic channels are still neglected. This is the level of simulation available in {\sc VBFNLO}.

All other codes ({\sc MG5\_aMC}, {\sc MoCaNLO+Recola}, {\sc PHANTOM}, and {\sc Whizard}) consider all contributions of order $\mathcal{O}{\left(\alpha^{6}\right)}$ as well as all possible interferences.
Note that the final W boson can always be considered either on-shell or off-shell without affecting the previous discussion.
All the codes mentioned here are described in details in the following sub-section.

Moving on to NLO accuracy, one can extend the approximations presented at LO.
The VBS approximation at NLO is straightforward for the virtual contributions, for the real-contributions one must be careful about gluon-initiated processes\footnote{The initial gluon must not couple to the other initial quark, otherwise there are infrared divergences proportional to $s$-channels which do not match with the ones found in the virtual contributions.
The subset of diagrams where all couplings of the initial state gluon to initial state quark are neglected forms a gauge-invariant subset, with the same argument as presented above. This approach is also fully consistent with the picture of two separate SU(3) copies.
}.
This is implemented in {\sc POWHEG}.
This approximation can be used in combination with a double-pole approximation \cite{Dittmaier:2015bfe} for the virtual contribution.
Such an approximation is implemented in {\sc Bonsay}.
In {\sc VBFNLO}, the $s$-channel contributions are available as well and can be
added on top of the VBS approximation. For the real emission diagrams, thereby
as simplification the gluon emission is fully modelled only for initial-state
radiation. The effect of final-state radiation together with the corresponding
virtual contributions is included as a $K$-factor. 

A further refinement is to consider the full real contributions as well as part of the virtual.
In particular one can consider only one-loop amplitudes where there is no gluon exchange between the quarks and assuming a cancellation of the infrared (IR) poles. \MP{True? I cannot remember exactly what is included}.
\MR{What about gauge invariance? A comment about that would be useful as well.}
Such predictions are provided by {\sc MG5\_aMC}.

Finally the full $\mathcal{O}{\left(\alphas \alpha^{6}\right)}$ computation consists also of EW corrections in the virtual as well as real corrections \cite{Biedermann:2017bss}.
Such predictions are provided by the combination {\sc MoCaNLO+Recola} as published in Ref.~\cite{Biedermann:2017bss}.

In Tab.~\ref{tab:wg1_codes} the details of the various codes are reported. In particular, it is specified whether
\begin{itemize}
    \item all $s$- and $t/u$-channel diagrams that lead to the considered final state are included;
    \item interferences between diagrams are included at LO;
    \item diagrams which do not feature two resonant vector bosons are included;
    \item the so-called non-factorisable (NF) QCD corrections, that is the corrections where (real or virtual) gluons are exchanged between different quark lines,
        are included;
    \item EW corrections to the $\mathcal O (\alpha^5\alphas)$ interference are included. These corrections are of the same order as the NLO QCD corrections to
        the  $\mathcal O (\alpha^6$) term.
\end{itemize}

\begin{table*}[ht!]
    \footnotesize
    \begin{tabularx}{\textwidth}{c|c|X|X|X|X|X}
        Code  &  $\mathcal O(\alpha^6)$ $|s|^2/$ $|t|^2/|u|^2$  &  $\mathcal O(\alpha^6)$ interf.  &  Non-res.  & NLO &  NF QCD  &  EW corr. to $\mathcal O(\alphas \alpha^5)$  \\
        \hline
        \hline
        {\sc Bonsay}        &  $t/u$    &  No       &  Yes, virt. No    &  Yes   & No       &  No  \\
        {\sc POWHEG}        &  $t/u$    &  No       &  Yes              &  Yes   & No       &  No  \\
        {\sc MG5\_aMC}      &  Yes      &  Yes      &  Yes              &  Yes   & No virt. &  No \\
        {\sc MoCaNLO+Recola}&  Yes      &  Yes      &  Yes              &  Yes   & Yes      &  Yes  \\
        {\sc PHANTOM}       &  Yes      &  Yes      &  Yes              &  No    & -        & - \\
        {\sc VBFNLO}        &  Yes      &  No       &  Yes              &  Yes   & No       &  No  \\
        {\sc Whizard}       &  Yes      &  Yes      &  Yes              &  No    & -        & - \\
    \end{tabularx}
    \caption{\label{tab:wg1_codes} Summary of the different properties of the codes employed in the comparison.}
\end{table*}
