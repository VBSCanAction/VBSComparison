We now discuss the various approximations which are implemented in computer programs for the EW contribution at order $\mathcal{O}{\left(\alpha^{6}\right)}$.
Since we are mostly interested in 
the scattering of two $\PW$ gauge bosons, which includes the quartic gauge-boson vertex, it may appear justified to approximate the 
full process by considering just those diagrams which contain the $2\rightarrow 2$ scattering process as a sub-part.
However, this set of contributions is not gauge invariant.
In order to ensure gauge invariance, an on-shell projection of
the incoming and outgoing W bosons should be performed.
While this can be done in the usual way for the time-like
outgoing W bosons, the treatment of the space-like W bosons
emitted from the incoming quarks requires some care.
Following Refs.~\cite{Kuss:1995yv,Accomando:2006hq} these W-boson lines can be split,
the W bosons entering the scattering process can be projected
on-shell, and the emission of the W bosons from the quarks can be
described by vector-boson luminosities.
Such an approximation is usually called effective vector-boson
approximation (EVBA) \cite{Dawson:1984gx,Duncan:1985vj,Cahn:1983ip}.

An improvement of such an approximation consists 
in considering all $t$- and $u$-channel diagrams and squaring them separately, neglecting interference contributions between the two classes.
These interferences are expected to be small in the VBS fiducial region, as they are both phase-space and colour suppressed~\cite{Oleari:2003tc,Denner:2012dz}.
The $s$-channel squared diagrams and any interferences between them and the  $t/u$-channels are also discarded.
This approximation is often called $t$-/$u$- approximation, VBF, or even VBS approximation.
We will adopt the latter denomination in the following of the article.
This approximation is gauge-invariant, a fact that can be appreciated by considering the two incoming quarks as belonging to two different copies of the $\rm{SU}(3)$ gauge group.

A further refinement is to add the squared matrix element of the $s$-channel contributions to the VBS approximation.

The approximations performed at LO can be extended when NLO QCD corrections to the order $\mathcal{O}{\left(\alpha^{6}\right)}$ are computed.
The VBS approximation can be extended at NLO in a straightforward manner for what concerns the virtual contributions.
For the real-emission contributions special care must be taken for the gluon-initiated processes.
The initial-state gluon and quark 
    must not couple together, otherwise infrared divergences proportional to $s$-channels will appear, 
     which do not match with the ones found in the virtual contributions.
     The subset of diagrams where all couplings of the initial state gluon to initial state quark are neglected forms a gauge-invariant subset, with the same argument presented above.

A further refinement is to consider the full real contributions, which include all interferences, and part of the virtual.
In particular one can consider only one-loop amplitudes where there is no gluon exchange between the two quark lines and 
assuming a cancellation of the infrared (IR) poles.

When considering the NLO corrections of order
$\mathcal{O}{\left(\alpha_{\rm s}\alpha^{6}\right)}$
to the full process, besides real and virtual QCD corrections
to the EW tree-level contribution of order
$\mathcal{O}{\left(\alpha^{6}\right)}$
also real and virtual EW corrections to the LO interference
of order $\mathcal{O}{\left(\alpha_{\rm s}\alpha^{5}\right)}$
have to be taken into account. Since some loop diagrams contribute
to both types of corrections, QCD and EW corrections cannot be
separated any more on the basis of Feynman diagrams, and the
cancellation of IR singularities requires the inclusion of all of them \cite{Biedermann:2017bss}.
