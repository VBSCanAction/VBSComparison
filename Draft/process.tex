As mentioned previously, the EW contribution is dominated by the scattering of two W gauge bosons.
Therefore it is justified to approximate the full EW contributions simply by the VBS contributions.
Nonetheless, this set of contributions is not gauge invariant.
To make it gauge-invariant, one should project on-shell the incoming W boson.
Unfortunately, this momentum is space-like and thus a simple on-shell projection is not possible.
Instead, one can keep the W boson leg connected to the external quark line off-shell while the one connected to the scattering is put on-shell.
Then the polarisation of the gauge boson is accommodated following the implementations of Refs.~\cite{Kuss:1995yv,Accomando:2006hq}.
Such an approximation is usually called effective vector-boson approximation (EVBA).

A more refined approximation consists in considering all $t$- and $u$- diagrams and square them separately.
Nonetheless, different type of diagrams are not squared which amounts to neglecting interferences.
These interferences are expected to be small in the VBS fiducial region.
The $s$-channel contributions can still be left out.
This approximation is often called $t$-/$u$- approximation or even VBS approximation.
We will adopt the latter denomination in the following of the article.
Such an approximation is implemented at LO in the computer codes {\sc Bonsay} and {\sc Powheg}.
The physical origin of this approximation is that each proton emitting a quark represents two independent copies of the $SU_c\left(3\right)$ gauge group.

The squared matrix element of the $s$-channel contributions can then be added but all interferences between different kinematic channels are neglected.
This is what is done in {\sc VBFNLO}.

All other codes ({\sc MG5\_aMC}, {\sc MoCaNLO+Recola}, {\sc PHANTOM}, and {\sc Whizard}) consider all contributions of order $\mathcal{O}{\left(\alpha^{6}\right)}$ as well as all possible interferences.
Note that the final W boson can always be considered either on-shell or off-shell without affecting the previous discussion.
All the codes mentioned here are described in details in the following sub-section.

Moving on to NLO accuracy, one can extend the approximation presented at LO.
The VBS approximation at NLO is simply extending the same approximation to real as well as to virtual corrections.
This is implemented in {\sc POWHEG}.
This approximation can be used in combination with a double-pole approximation [REF] for the virtual contribution.
This requires the computation of factorisable as well as non-factorisable corrections \cite{Dittmaier:2015bfe} separately.
Such an approximation is implemented in {\sc Bonsay}.
In {\sc VBFNLO}, the VBS approximation at NLO with $s$-channel contributions is implemented.

A further refinement is to consider the full real contributions as well as part of the virtual.
In particular one can consider only one-loop amplitudes where there is no gluon exchange between the quarks and assuming a cancellation of the infrared (IR) poles. \MP{True? I cannot remember exactly what is included}.
Such predictions are provided by {\sc MG5\_aMC}.

Finally the full $\mathcal{O}{\left(\alphas \alpha^{6}\right)}$ computation consists also of EW corrections in the virtual as well as real corrections \cite{Biedermann:2017bss}.
Such predictions are provided by the combination {\sc MoCaNLO+Recola} as published in Ref.~\cite{Biedermann:2017bss}.

In Tab.~\ref{tab:wg1_codes} the details of the various codes are reported. In particular, it is specified whether
\begin{itemize}
    \item all $s$- and $t/u$-channel diagrams that lead to the considered final state are included;
    \item interferences between diagrams are included at LO;
    \item diagrams which do not feature two resonant vector bosons are included;
    \item the so-called non-factorisable (NF) QCD corrections, that is the corrections where (real or virtual) gluons are exchanged between different quark lines,
        are included;
    \item EW corrections to the $\mathcal O (\alpha^5\alphas)$ interference are included. These corrections are of the same order as the NLO QCD corrections to
        the  $\mathcal O (\alpha^6$) term.
\end{itemize}

\begin{table*}[ht!]
    \footnotesize
    \begin{tabularx}{\textwidth}{c|c|X|X|X|X|X}
        Code  &  $\mathcal O(\alpha^6)$ $|s|^2/$ $|t|^2/|u|^2$  &  $\mathcal O(\alpha^6)$ interf.  &  Non-res.  & NLO &  NF QCD  &  EW corr. to $\mathcal O(\alphas \alpha^5)$  \\
        \hline
        \hline
        {\sc Bonsay}        &  $t/u$    &  No       &  Yes, virt. No    &  Yes   & No       &  No  \\
        {\sc POWHEG}        &  $t/u$    &  No       &  Yes              &  Yes   & No       &  No  \\
        {\sc MG5\_aMC}      &  Yes      &  Yes      &  Yes              &  Yes   & No virt. &  No \\
        {\sc MoCaNLO+Recola}&  Yes      &  Yes      &  Yes              &  Yes   & Yes      &  Yes  \\
        {\sc PHANTOM}       &  Yes      &  Yes      &  Yes              &  No    & -        & - \\
        {\sc VBFNLO}        &  Yes      &  No       &  Yes              &  Yes   & No       &  No  \\
        {\sc Whizard}       &  Yes      &  Yes      &  Yes              &  No    & -        & - \\
    \end{tabularx}
    \caption{\label{tab:wg1_codes} Summary of the different properties of the codes employed in the comparison.}
\end{table*}
