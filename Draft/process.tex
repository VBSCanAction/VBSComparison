Describe the physics and mention the code that simulate these cases. \\
Start with LO (one paragraph each). \\

- Details on the description starting from the VBS approximation which we define as the t-u approximation (other names in the litterature) \\
Start from the idea of two independent protons etc. (POWHEG) \\
- adding $s$-channel contributions, explain why this is possible to add them separately (VBFNLO)\\
- Full computation (MG, MoCaNLO-Recola, Phantom) \\

Move to NLO (one paragraph each). \\

- VBS approximation at NLO (POWHEG) \\
- VBS approximation at NLO + DPA for virt (Bonsay) \\
- VBS approximation + $s$-channel (VBS NLO) \\
- Hybrid VBS approximation (MG) \\
- Explanation why EW corrections are needed in the full computation (Recola) \\

\MP{Part written by Giovanni to be included}

The VBS approximation [?] is frequently employed for VBS computations and we aim at the identification of kinematical regions where it provides trustworthy prediction for the $W^+W^+$ scattering.
At LO, given the full set of diagrams contributing at order $\mathcal{O}(\alpha_{ew}^6)$, the approximations consists in:
\begin{itemize}
\item discarding interferences between $t$ and $u$ channel diagrams, which are expected to be suppressed in the fiducial volume, after VBF cuts;
\item discarding $s$--channel diagrams shown in \autoref{fig:jjpeak_diag}, which contain $q\bar{q}'$ annihilations ($W^-\rightarrow q \bar{q}'$); with a hard cut on the $jj$--pair invariant mass, these contributions are strongly suppressed.
\end{itemize}
