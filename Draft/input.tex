The hadronic scattering processes are simulated at the LHC with a center-of-mass energy $\sqrt s = 13 \TeV$.
The NNPDF~3.0 parton distribution functions~(PDFs)~\cite{Ball:2014uwa} with five flavour scheme, NLO-QCD evolution, and a strong coupling constant $\alphas\left( \MZ \right) = 0.118$ are employed.\footnote{The corresponding identifier {\tt lhaid} in LHAPDF6~\cite{Buckley:2014ana} is 260000.}
Since this PDF set has no photonic density, photon-induced processes are not considered. \BJ{I would change that sentence to simply {\em Photon-induced processes are not considered.}, as there are more reasons for disregarding the photon-induced contributions. } 
Initial-state collinear singularities are factorised according to the ${\overline{\rm MS}}$ scheme, consistently with what is done in NNPDF. \BJ{Should we say something about singularities due to soft photon exchange in the $t$-channel that are removed in VBFNLO and POWHEG by a technical cutoff?}

For the massive particles, the following masses and decay widths are used:
%
\begin{alignat}{2}
                  \Mt   &=  173.21\GeV,       & \quad \quad \quad \Gt &= 0 \GeV,  \nonumber \\
                \MZOS &=  91.1876\GeV,      & \quad \quad \quad \GZOS &= 2.4952\GeV,  \nonumber \\
                \MWOS &=  80.385\GeV,       & \GWOS &= 2.085\GeV,  \nonumber \\
                M_{\rm H} &=  125.0\GeV,       &  \GH   &=  4.07 \times 10^{-3}\GeV.
\end{alignat}
%
The measured on-shell (OS) values for the masses and widths of the W and Z bosons are converted into pole values for the gauge bosons ($V=\PW,\PZ$) according to Ref.~\cite{Bardin:1988xt},
%
\begin{equation}
\begin{split}
        M_V &= \MVOS/\sqrt{1+(\GVOS/\MVOS)^2}\,, \\
   \Gamma_V &= \GVOS/\sqrt{1+(\GVOS/\MVOS)^2}.
\end{split}
\end{equation}
%
The EW coupling is renormalised in the $G_\mu$ scheme \cite{Denner:2000bj}, where
%
\begin{equation}
    G_{\mu}    = 1.16637\times 10^{-5}\GeV^{-2}.
\end{equation}
%
The numerical value of $\alpha$, corresponding to the choice of input parameters is
%
\begin{equation}
 \alpha = 7.555310522369 \times 10^{-3}.
\end{equation}
The CKM-Matrix is assumed to be diagonal, meaning that the mixing between different quark families is neglected.
The complex-mass scheme~\cite{Denner:1999gp,Denner:2005fg,Denner:2006ic} is used throughout to treat unstable intermediate particles in a gauge-invariant manner.

The renormalisation and factorisation scales are set to the dynamical scale
%
\begin{equation}
\label{eq:defscale}
 \mu_{\rm ren} = \mu_{\rm fac} = \sqrt{p_{\rm T, j_1}\, p_{\rm T, j_2}}, 
\end{equation}
%
\BJ{defined via the transverse momenta of the two hardest identified jets. } 
This choice of scale has been shown to provide stable NLO-QCD predictions \cite{Denner:2012dz}.

Following experimental measurements \cite{Aad:2014zda,Aaboud:2016ffv,Khachatryan:2014sta,CMS:2017adb}, the event selection used in the present study is:

\begin{itemize}
    \item The two same-sign charged leptons are required to \BJ{fulfill cuts on transverse momentum, rapidity, and separation in the rapidity-azimuthal angle separation, }
        \begin{align}
        \label{cut:1}
         \ptsub{\Pl} >  20\GeV,\qquad |y_{\Pl}| < 2.5, \qquad \Delta R_{\Pl\Pl}> 0.3\,.
        \end{align}
        %
    \item The total missing transverse energy, computed from the vectorial sum of the transverse momenta of the two neutrinos, is required to be
        \begin{align}
        \label{cut:2}
          \etsub{\text{miss}}=p_{\rm T, miss} >  40\GeV\,.
        \end{align}
        %
    \item QCD partons (quarks and gluons) are clustered together using the anti-$k_T$ algorithm~\cite{Cacciari:2008gp} with distance parameter $R=0.4$. %Jets are required to have
    \BJ{We impose cuts on the jets' transverse momenta, rapidities, and their separation from leptons, } 
        \begin{align}
        \label{cut:3}
         \ptsub{\Pj} >  30\GeV, \qquad |y_\Pj| < 4.5, \qquad \Delta R_{\Pj\Pl} > 0.3 \,.
        \end{align}
        VBS cuts are applied on the two jets with largest transverse-momentum, unless otherwise stated. In particular, we impose a cut on the 
         in\-vari\-ant mass of the di-jet system,  as well as on the rapidity-separation of the two jets,          \begin{align}
        \label{cut:4}
         m_{\Pj \Pj} >  500\GeV,\qquad |\Delta y_{\Pj \Pj}| > 2.5, 
        \end{align}
        if not explicitly stated otherwise. 
        %
    \item When EW corrections are computed, real photons and charged fermions are clustered together using the anti-$k_T$ algorithm with
        radius parameter $R=0.1$. In this case, leptons and quarks are understood as {\it dressed fermions}.
\end{itemize}
