We simulate VBS production at the LHC, with a center-of-mass energy $\sqrt s = 13 \TeV$. We assume five massless flavours in the proton, and employ the NNPDF~3.0 parton
density~\cite{Ball:2014uwa}
with NLO QCD evolution (the {\tt lhaid} in LHAPDF6~\cite{Buckley:2014ana} for this set is 260000) and strong coupling constant $\alphas\left( \MZ \right) = 0.118$. Since
the employed PDF set has no photonic density, photon-induced processes are not considered. Initial-state collinear singularities are factorised with the  ${\overline{\rm MS}}$
scheme, consistently with what is done in NNPDF.\\
We use the following values for the mass and width of the massive particles:
%
\begin{alignat}{2}
                  \Mt   &=  173.21\GeV,       & \quad \quad \quad \Gt &= 0 \GeV,  \nonumber \\
                \MZOS &=  91.1876\GeV,      & \quad \quad \quad \GZOS &= 2.4952\GeV,  \nonumber \\
                \MWOS &=  80.385\GeV,       & \GWOS &= 2.085\GeV,  \nonumber \\
                M_{\rm H} &=  125.0\GeV,       &  \GH   &=  4.07 \times 10^{-3}\GeV,
\end{alignat}
and renormalise the EW coupling in the $G_\mu$ scheme \cite{Denner:2000bj} where
\begin{equation}
    G_{\mu}    = 1.16637\times 10^{-5}\GeV^{-2}.
\end{equation}
The derived value of the EW coupling $\alpha$, corresponding to our choice of input parameters, is
\begin{equation}
 \alpha = 7.555310522369 \times 10^{-3}. \\
\end{equation}
We employ the complex-mass scheme~\cite{Denner:1999gp,Denner:2005fg} to treat unstable intermediate particles in a gauge-invariant manner {\bf CHECK THAT ALL CODES USE THE CMS}.\\

The renormalisation and factorisation scales are set dynamically as
%
\begin{equation}
\label{eq:defscale}
 \mu_{\rm ren} = \mu_{\rm fac} = \sqrt{p_{\rm T, j_1}\, p_{\rm T, j_2}},
\end{equation}

Cross sections and distribution are computed within the following VBS cuts inspired from experimental measurements \cite{Aad:2014zda,Aaboud:2016ffv,Khachatryan:2014sta,CMS:2017adb}:
\begin{itemize}
    \item The two same-sign charged leptons are required to have
        \begin{align}
         \ptsub{\Pl} >  20\GeV,\qquad |y_{\Pl}| < 2.5, \qquad \Delta R_{\Pl\Pl}> 0.3\,.
        \end{align}
    \item The total missing transverse energy, computed from the vectorial sum of the transverse momenta of the two neutrinos in the event,
        is required to be
        \begin{align}
          \etsub{\text{miss}}=p_{\rm T, miss} >  40\GeV\,.
        \end{align}
    \item QCD partons (quarks and gluons) are clustered together using the anti-$k_T$ algorithm~\cite{Cacciari:2008gp} with distance parameter $R=0.4$. Jets are required
        to have
        \begin{align}
         \ptsub{\Pj} >  30\GeV, \qquad |y_\Pj| < 4.5, \qquad \Delta R_{\Pj\Pl} > 0.3 \,.
        \end{align}
        On the two jets with largest transverse-momentum the following invariant-mass and rapidity-separation cuts are imposed
        \begin{align}
         m_{\Pj \Pj} >  500\GeV,\qquad |\Delta y_{\Pj \Pj}| > 2.5.
        \end{align}
%         Finally, all jest in the event are required to be separated from charged leptons:
%         \begin{align}
%          \qquad\Delta R_{\Pj\Pl} > 0.3 .
%         \end{align}
    \item When EW corrections are computed, real photons and charged fermion are clustered together using the anti-$k_T$ algorithm with
        radius parameter $R=0.1$. In this case, leptons and quarks mentioned above must be understood as {\it dressed fermions}. Photons
        which are not combined at this step are clustered with QCD partons to form jets as it is described previously.
\end{itemize}
