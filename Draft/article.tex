%% Main-Version Draft
\documentclass[twocolumn,epjc3]{svjour3} % use the EPJC template, two columns
%%\documentclass[epjc3]{svjour3} % use the EPJC template, one column
\smartqed  % flush right qed marks, e.g. at end of proof
%% -------------------------------
%% |          Packages           |
%% -------------------------------
 \usepackage{amsmath}
 \usepackage{amssymb}
 \usepackage{graphicx}
  \usepackage[merge,numbers,compress]{natbib}
 \usepackage[T1]{fontenc}
 \usepackage{booktabs}
 \usepackage{xcolor}
 \usepackage{xspace}
 \usepackage{dcolumn}
%  \usepackage{hyperref}
  \usepackage[draft]{hyperref}
 \usepackage{caption}
 \usepackage{tabularx}
 \usepackage{lmodern}
 \usepackage{url}
 \usepackage{morefloats}
 \usepackage[utf8]{inputenc}
%  \usepackage{siunitx}
% \usepackage
% [subrefformat=parens,position=top,skip=-15pt,margin=15pt,justification=justified,singlelinecheck=false]
% {subcaption}
\hyphenation{quan-ti-ta-tive-ly fer-mi-on fer-mi-ons}

%\setlength{\evensidemargin}{0cm}
%\setlength{\oddsidemargin}{0cm}
%\setlength{\topmargin}{0.00cm}
%\setlength{\textwidth}{16.0cm}
%\setlength{\textheight}{22.55cm}
%\setlength{\headheight}{0cm}
%\setlength{\headsep}{0cm}
%\setlength{\voffset}{0cm}
%\setlength{\paperheight}{27cm}

%\renewcommand{\topfraction}{0.8}
%\renewcommand{\bottomfraction}{0.5}
%\renewcommand{\textfraction}{0.2}
%\renewcommand{\floatpagefraction}{0.7}

\newcommand{\MP}[1]{{ {\color{blue}{ [MP: #1]}} }}
\newcommand{\MR}[1]{{ {\color{red}{ [MR: #1]}} }}
\newcommand{\AK}[1]{{ {\color{magenta}{ [AK: #1]}} }}
\newcommand{\ZM}[1]{{ {\color{green}{ [MZ: #1]}} }}

\newcommand{\BJ}[1]{{ {\color{cyan}{ [BJ: #1]}} }}



%% ---------------------------------
%% | ToDo Marker - only for draft! |
%% ---------------------------------
% Remove this section for final version!
%\setlength{\marginparwidth}{20mm}

\newcommand{\margtodo}
{\marginpar{\textbf{\textcolor{red}{ToDo}}}{}}

\newcommand{\todo}[1]
{{\textbf{\textcolor{red}{(\margtodo{}#1)}}}{}}

\input{macros}
%
\journalname{DESY 18-025, FR-PHENO-2018-003, KA-TP-05-2018, ZU-TH-08/18} % Preprint numbers here
%
\begin{document}

\title{Precise predictions for same-sign W-boson scattering at the LHC}

\author{Alessandro Ballestrero\thanksref{infnto}
\and
Benedikt Biedermann\thanksref{wurz}
\and
Simon Brass\thanksref{sieg}
\and
Ansgar Denner\thanksref{wurz}
\and
Stefan Dittmaier\thanksref{frei}
\and
Pietro Govoni\thanksref{mila}
\and
Michele Grossi\thanksref{pavia,ibm}
\and
Barbara J\"ager\thanksref{tub}
\and
Alexander Karlberg\thanksref{uzh}
\and
Ezio Maina\thanksref{infnto,unito}
\and
Mathieu Pellen\thanksref{wurz}
\and
Giovanni Pelliccioli\thanksref{infnto,unito}
\and
Simon Pl\"atzer\thanksref{wien}
\and
Michael Rauch\thanksref{kit}
\and
Daniela Rebuzzi\thanksref{pavia}
\and
J\"urgen Reuter\thanksref{desy}
\and
Vincent Rothe\thanksref{desy}
\and
Christopher Schwan\thanksref{frei}
\and
Pascal Stienemeier\thanksref{desy}
\and
Giulia Zanderighi\thanksref{cern}
\and
Marco Zaro\thanksref{nikh}
\and
Dieter Zeppenfeld\thanksref{kit}
}


% \thankstext{e1}{e-mail: mpellen@physik.uni-wuerzburg.de}
% \thankstext{e2}{e-mail: m.zaro@nikhef.nl}

\institute{
INFN, Sezione di Torino, %
Via P. Giuria 1, %
10125 Torino, %
Italy\label{infnto}
\and
Universit\"at W\"urzburg, %
Institut f\"ur Theoretische Physik und Astrophysik,  %
Emil-Hilb-Weg 22, %
97074 W\"urzburg, %
Germany\label{wurz}
\and
Universit\"at Siegen, Department Physik, %
 Walter-Flex-Str.3, %
57068 Siegen, Germany\label{sieg}
\and
Albert-Ludwigs-Universit\"at Freiburg, Physikalisches Institut, %
 Hermann-Herder-Str.\ 3, %
79104 Freiburg, Germany\label{frei}
\and
Milan, Italy\label{mila}
\and
Universit\'a di Pavia, Dipartimento di Fisica and INFN, Sezione di Pavia, %
Via A. Bassi 6, %
27100 Pavia, %
Italy\label{pavia}
\and
IBM Italia s.p.a. %
Circonvallazione Idroscalo , %
20090 Segrate (MI), %
Italy\label{ibm}
\and
Institute for Theoretical Physics,
University of T\"ubingen,
Auf der Morgenstelle 14,
72076 T\"ubingen,
Germany\label{tub}
\and
Physik-Institut,
Universit\"at Z\"urich,
Winterthurerstrasse 190,
CH-8057 Z\"urich,
Switzerland\label{uzh}
\and
Universit\`a di Torino, Dipartimento di Fisica, %
Via P. Giuria 1, %
10125 Torino, %
Italy\label{unito}
\and
Particle Physics, Faculty of Physics, University of Vienna, %
Vienna, %
Austria\label{wien}
\and
Institute for Theoretical Physics, Karlsruhe Institute of
Technology (KIT), %
76131 Karlsruhe,
Germany\label{kit}
\and
DESY Theory Group, %
Notkestr. 85, %
22607 Hamburg,
Germany\label{desy}
\and
CERN,
Theoretical Physics Department,
CH-1211, Geneva 23,
Switzerland\label{cern}
\and
Nikhef,
Science Park 105,
1098XG Amsterdam,
The Netherlands\label{nikh}
}

\maketitle
% \AK{Some general comments: We don't write the names of programs consistently everywhere. I think the nicest is to use}
% \begin{verbatim} 
% {\sc Generator}
% \end{verbatim}
% \AK{Ie captial first letter and lower-case everywhere else for generator names instead of all capital. But this is a matter of taste of course...}
% \AK{We may want to make it more clear in the abstract/introduction why this study is worth publishing. So focus on the large contributions from non-VBS approxiamtion at NLO which until now has been reproted as small, and the large discrepancies in the NLO+PS predictions. That plus the recommendations we give should be the main selling point, I think.}
\begin{abstract}
\noindent

In this article, a detailed study of the vector-boson scattering for two positively charged W bosons is presented.
In particular, a comparison between the full next-to-leading (NLO) QCD corrections against several approximations is carried out.
This study is not only performed in the usual fiducial region used by experimental collaborations but also in a more inclusive set-up.
This allows to infer precisely the quality of such approximations.
Finally, NLO predictions matched to various parton shower are also discussed.
Thanks to this, it is thus possible to infer the systematic errors related to vector-boson scattering at the NLO-QCD level and beyond.

\end{abstract}
\thispagestyle{empty}
\vfill
\newpage
\setcounter{page}{1}

%\tableofcontents
\newpage


\section{Introduction}
Vector-boson scattering (VBS) at a hadron collider 
usually refers to the interaction of massive vector-bosons ($\PW^\pm,\PZ$),
irradiated by partons (quarks) of the incoming protons, 
which in turn are deflected from the beam direction 
and enter the volume of the particle detectors.
As a consequence, the typical signature of VBS events
is characterised by two energetic jets 
and four fermions,
originating from the decay of the two vector bosons.
Among the possible diagrams,
the scattering process can be mediated also by a Higgs boson
and involve in particular their longitudinal component.
The interaction of of longitudinally polarised bosons is of particular interest, 
because the corresponding matrix elements feature both gauge and unitarity cancellations 
that strongly depend on the actual structure of the Higgs sector.
A detailed study of this class of processes will therefore further constrain the Higgs couplings 
at a very different energy scale with respect to the Higgs boson mass,
and hint at, or exclude, non-Standard Model behaviours.

The VBS process involving two same-sign $\PW$ bosons has the largest signal-to-background ratio at the LHC:
evidence for it was found at the centre-of-mass energy of $8~\TeV$ already~\cite{Aad:2014zda,Khachatryan:2014sta},
and it has been recently observed~\cite{Sirunyan:2017ret} and measured~\cite{Aaboud:2016ffv} 
at $13\TeV$ as well.
Presently, the measurements of VBS processes are limited by statistics, but the situation will change in a near future.
On the theoretical side, 
it is thus of prime importance to provide predictions with systematic uncertainties
at least comparable to the current and experimental precisions~\cite{CMS:2016rcn}.


The $\PW^+\PW^+$ scattering is also the simplest VBS process to calculate, 
because the double-charge structure of the leptonic final state 
limits the number of partonic processes and total number of Feynman diagrams for each process.
Therefore, this process is the ideal candidate for a comparative study of the different simulation tools.

In the last few years, several next-to-leading order (NLO) computations became available for both the VBS process~\cite{Jager:2006zc,Jager:2006cp,Bozzi:2007ur,Jager:2009xx,Jager:2011ms,Denner:2012dz,Rauch:2016pai} and its QCD-induced irreducible background process~\cite{Rauch:2016pai,Melia:2010bm,Melia:2011gk,Campanario:2013gea,Baglio:2014uba}.
The VBS computations all rely on approximations, while recently the complete NLO corrections have been performed~\cite{Biedermann:2017bss}.
It is therefore interesting to infer in details the quality of the various approximations.
Indeed, apart from Ref.~\cite{Biedermann:2017bss} where it is commented on, \MP{more references?} no detailed comparison of the VBS approximations have been carried out.
Preliminary results of the present study have already been made public in Ref.~\cite{Anders:2018gfr}.

The full gauge-invariant process including the $\PW^+\PW^+$ scattering 
 is $\Pp\Pp\to\mu^+\nu_\mu{\rm e}^+\nu_{\rm e}\,\Pj\Pj+\mathrm{X}$.
This final state receives three contributions at leading order (LO) whose coupling orders are $\mathcal{O}{\left(\alpha^{6}\right)}$, $\mathcal{O}{\left(\alpha_{\rm s}\alpha^{5}\right)}$, and $\mathcal{O}{\left(\alphas^{2}\alpha^{4}\right)}$.
They are commonly referred to as electroweak (EW), interference, and QCD contributions, respectively.%
\footnote{The EW contribution is sometimes referred to as the VBS contribution, even it involves also non-VBS contributions.}
Therefore, the present work starts with a LO study of these three contributions as a function of typical VBS cuts.
This allows to quantify the various contributions to the final state $\mu^+\nu_\mu{\rm e}^+\nu_{\rm e}\,\Pj\Pj$.
This is followed by a LO comparison between the various predictions at the level of the cross section and differential distributions.

At NLO, the process possesses four contributions of orders $\mathcal{O}{\left(\alpha^{7}\right)}$, $\mathcal{O}{\left(\alphas\alpha^{6}\right)}$, $\mathcal{O}{\left(\alphas^{2}\alpha^{5}\right)}$, and $\mathcal{O}{\left(\alphas^{3}\alpha^{4}\right)}$.
The largest ones are the EW corrections~\cite{Biedermann:2017bss,Biedermann:2016yds} of order $\mathcal{O}{\left(\alpha^{7}\right)}$.
The contribution to the order $\mathcal{O}{\left(\alphas\alpha^{6}\right)}$ is the second largest NLO contribution and is often referred to as the QCD corrections to the VBS process.
In the following, this order is the one where our comparisons are focused on and we will refer to it as simply \emph{NLO}.
As for the LO study, the various predictions are compared at the level of the cross section and differential distributions now at NLO accuracy.
In particular, this makes it possible to infer the accuracy of the so-called VBS approximation, which we will define in more details later.
To our knowledge, such a detailed study was still missing.

Finally, several predictions featuring parton shower are compared, giving the possibility to infer systematic differences between the various predictions.
A first study for the $\PW^+ \PW^- \Pj \Pj$ VBS process has been presented in
Ref.~\cite{Rauch:2016upa}, comparing the angular-ordered default shower and the dipole
shower and both{\sc MC@NLO}-like~\cite{Frixione:2002ik} and {\sc POWHEG}-like~\cite{Nason:2004rx,Frixione:2007vw} matching as implemented
in {\sc Herwig 7}~\cite{Bellm:2015jjp}, also showing scale-variation uncertainties. In this work we extend such a comparison by presenting 
results obtained by {\sc MadGraph5\_aMC@NLO}, {\sc Powheg} and {\sc Phantom} {\bf CITES}, possibly matched with different parton-showers.
This is the first time in the literature that NLO QCD calculations for VBS processed matched to parton shower are compared between different generators.

The article is organised as follows:
first, we define the process under study in Sec.~\ref{sec:definition}.
The various approximation which are provided by the different computer codes at LO and NLO are described in Sec.~\ref{sec:details}.
This is followed by a presentation of the programs used for the computations.
Sections~\ref{sec:LO} and \ref{sec:NLO} are devoted to a LO and NLO study at fixed order, respectively.
Section~\ref{sec:matching} complements our study by comparing predictions at LO and LO which include the effect of parton showers and hadronization.
The last section consists in concluding remarks and recommendations for experimental collaborations.


\section{Definition of the process}
    \label{sec:definition}
    \begin{figure*}[t]
\begin{center}
          \includegraphics[width=0.30\linewidth]{feynman/LO_EW_5}
          \raisebox{.5ex}{\includegraphics[width=0.35\linewidth]{feynman/LO_EW_2}}
          \raisebox{-1.8ex}{\includegraphics[width=0.32\linewidth]{feynman/LO_EW_3}}
\end{center}
        \caption{Sample tree-level diagrams that contribute to the process $\Pp\Pp\to\mu^+\nu_\mu\Pe^+\nu_{\Pe}\Pj\Pj$ at order $\mathcal{O}{\left(\alpha^{6}\right)}$.
        In addition to typical VBS contribution (left), this order also possesses $s$-channel contributions such as decay chain (middle) and tri-boson contributions (right).}
\label{diag:LO}
\end{figure*}

The scattering of two positively-charged $\PW$ bosons with their subsequent decay into different-flavour leptons 
can proceed at the LHC through the partonic process:
%
\begin{equation}
\Pp\Pp\to\mu^+\nu_\mu{\rm e}^+\nu_{\rm e}\,\Pj\Pj+\mathrm{X}.
\end{equation}

This process possesses three LO contributions of different orders.
At LO, this process can proceed via three different coupling-order combinations:
$\mathcal{O}{\left(\alpha^{6}\right)}$, $\mathcal{O}{\left(\alphas{2}\alpha^{4}\right)}$ and $\mathcal{O}{\left(\alphas\alpha^{5}\right)}$.
The first, commonly referred to as EW contribution or VBS~\footnote{The name VBS is used even though not all Feynman diagrams involve the scattering
of vector bosons}, receives the contributions from Feynman diagrams such as those in Fig.~\ref{diag:LO}:
in addition to genuine VBS contributions (left diagram), it also features $s$-channel contributions with non-resonant vector bosons 
(center diagram) or from
triple-boson production (right diagram).
Note that $s$-, $t$-, and $u$-channel contributions are defined according to the quark lines. $s$-channel 
denotes all Feynman diagrams where the two initial-state partons are connected by a continuous fermion line, 
and $u$-channel refers to contributions with crossed fermion lines, which appears for identical quarks or anti-quarks in the final state
The $s$-channel contributions will play a particular role in the study of the various contributions in Sec.~\ref{subsec:contributions}.

When using approximations, care must be taken that only gauge-invariant subsets are considered to obtain physically meaningful results. We will discuss the commonly-used possible choices in detail in the next section.

The second coupling combination corresponds to diagrams with a gluon connecting the two quark lines, and with the $\PW$ bosons 
radiated off the quark lines. Because of the different colour structure, this contribution features
different kinematical behaviours than VBS. Nonetheless they share the same final state and therefore constitute an irreducible background.

Finally, the third contribution is the interference of diagrams belonging to the first with those belonging to the second. It is non-zero only for
those partonic subprocesses which involve only one quark family. Such a contribution is typically small but not negligible for realistic experimental set-ups \cite{Biedermann:2017bss}.

In experimental measurements, special cuts, called VBS cuts, are designed to enhance the EW contribution over the QCD one and to suppress the interference.
These cuts are based on the different kinematical behaviour of the two contributions.
The EW contribution is characterised by two jets with large rapidities as well as a large invariant mass.
The two $\PW$ bosons are mostly produced centrally.
This is in contrast to the QCD contribution which favours jets in the central region.
Therefore, the event selection usually involves rapidity-difference and invariant-mass cuts for the jets.
Note that, as pointed out in Ref.~\cite{Biedermann:2017bss}, when considering full amplitudes, the separation between EW and QCD production becomes ill
defined.
Hence, combined measurements which are better theoretically defined should be preferably performed by the experimental collaborations at the LHC.




\section{Details of the calculations}
    \label{sec:details}
    \subsection{Several descriptions for one process}
        As mentioned previously, the EW contribution is dominated by the scattering of two W gauge bosons.
Therefore it is justified to approximation the full EW contributions simply 


Describe the physics and mention the code that simulate these cases. \\
Start with LO (one paragraph each). \\

- Details on the description starting from the VBS approximation which we define as the t-u approximation (other names in the litterature) \\
Start from the idea of two independent protons etc. (POWHEG) \\
- adding $s$-channel contributions, explain why this is possible to add them separately (VBFNLO)\\
- Full computation (MG, MoCaNLO-Recola, Phantom) \\

Move to NLO (one paragraph each). \\

- VBS approximation at NLO (POWHEG) \\
- VBS approximation at NLO + DPA for virt (Bonsay) \\
- VBS approximation + $s$-channel (VBS NLO) \\
- Hybrid VBS approximation (MG) \\
- Explanation why EW corrections are needed in the full computation (Recola) \\

\MP{Part written by Giovanni to be included}

The VBS approximation [?] is frequently employed for VBS computations and we aim at the identification of kinematical regions where it provides trustworthy prediction for the $W^+W^+$ scattering.
At LO, given the full set of diagrams contributing at order $\mathcal{O}(\alpha_{ew}^6)$, the approximations consists in:
\begin{itemize}
\item discarding interferences between $t$ and $u$ channel diagrams, which are expected to be suppressed in the fiducial volume, after VBF cuts;
\item discarding $s$--channel diagrams shown in \autoref{fig:jjpeak_diag}, which contain $q\bar{q}'$ annihilations ($W^-\rightarrow q \bar{q}'$); with a hard cut on the $jj$--pair invariant mass, these contributions are strongly suppressed.
\end{itemize}

    \subsection{Description of the predictions}
        \label{subsec:codedescr}
        In the following, the codes employed throughout this paper and the approximations implemented in each of them will be discussed:

\begin{itemize}

\item The program {\sc Bonsay} consists of a general-purpose Monte Carlo integrator written by Christopher Schwan and matrix elements taken from different sources:
Born matrix elements are adapted from the program {\sc Lusifer}~\cite{Dittmaier:2002ap}, real matrix elements are written by Marina Billoni, and virtual matrix elements by Stefan Dittmaier.
One loop integrals are evaluated using the {\sc Collier} library~\cite{Denner:2014gla,Denner:2016kdg}.
For the fiducial cross sections it uses the VBS approximation at LO and NLO.
The virtual corrections are additionally approximated using a double-pole approximation.
For more inclusive cross sections at LO the exact matrix elements ($s$-channels, interferences) can also be used.

  \item {\sc MadGraph5\_aMC@NLO}~\cite{Alwall:2014hca} (henceforth {\sc MG5\_aMC}) is an automatic meta-code (a code that generates codes) which makes it possible to simulate any scattering process
      including NLO QCD corrections both at fixed order and including matching to parton showers, using the {\sc MC@NLO}\ method~\cite{Frixione:2002ik}. It makes use of the FKS subtraction method~\cite{Frixione:1995ms,
        Frixione:1997np} (automated in the module {\sc MadFKS}~\cite{Frederix:2009yq,
        Frederix:2016rdc}) for regulating IR singularities. The computations of one-loop amplitudes are carried out by switching dynamically between
        two integral-reduction techniques, OPP~\cite{Ossola:2006us} or Laurent-series expansion~\cite{Mastrolia:2012bu},
        and tensor-integral reduction~\cite{Passarino:1978jh,Davydychev:1991va,Denner:2005nn}. These have been automated in the module {\sc MadLoop}~\cite{Hirschi:2011pa}, which
        in turn exploits {\sc CutTools}~\cite{Ossola:2007ax}, {\sc Ninja}~\cite{Peraro:2014cba,
        Hirschi:2016mdz}, {\sc IREGI}~\cite{ShaoIREGI}, or {\sc Collier}~\cite{Denner:2016kdg}, together with an in-house 
        implementation of the {\sc OpenLoops} optimisation~\cite{Cascioli:2011va}. Finally, scale and PDF uncertainties can be obtained in an exact manner via reweighting
        at zero additional CPU cost~\cite{Frederix:2011ss}.\\
        The simulation of VBS at NLO-QCD accuracy can be performed by issuing the following commands in the program interface:
\begin{verbatim}
> set complex_mass_scheme
> import model loop_qcd_qed_sm_Gmu
> generate p p > e+ ve mu+ vm j j QCD=0 [QCD]
> output
\end{verbatim}
  With these commands the complex-mass scheme is turned on, then the NLO-capable model is loaded\footnote{Despite
            the {\tt loop\_qcd\_qed\_sm\_Gmu} model also includes NLO counterterms for computing EW corrections, it is not yet possible to compute such corrections
        with the current public version of the code.}, finally the process code is generated (note the {\tt QCD=0} syntax to select the purely-EW process)
        and written to disk. No approximation is performed for the Born and real-emission matrix elements. 
        For what concerns the virtual matrix element, because of some internal limitations which will be lifted in the future version capable of computing both QCD and EW corrections,
        only loops with QCD-interacting particles are generated. Such an approximation is equivalent to the assumption that the finite part of
        those loops which feature EW bosons is zero. In practice, since a part of the contribution to the single pole is also missing, the internal 
        pole-cancellation check at run time has to be turned off, by setting the value of the {\tt IR\-Pole\-Check\-Threshold} and 
        {\tt Precision\-Virtual\-At\-Run\-Time} parameters in the {\tt Cards\-/FKS\_\-params.dat} file to -1.

\item The program {\sc MoCaNLO+Recola} is made of a flexible Monte Carlo program dubbed {\sc MoCaNLO} and of the matrix element generator {\sc Recola}~\cite{Actis:2012qn,Actis:2016mpe}.
It can compute arbitrary processes for the LHC at both NLO QCD and EW accuracy in the Standard Model.
This is made possible by the fact that {\sc Recola} can compute arbitrary processes at tree and one-loop level in the Standard Model.
To that end, it relies on the {\sc Collier} library \cite{Denner:2014gla,Denner:2016kdg} to numerically evaluate the one-loop scalar and tensor integrals.
In addition, the subtraction of the IR divergences appearing in the real corrections has been automatised thanks to the Catani--Seymour dipole formalism for both QCD and QED \cite{Catani:1996vz,Dittmaier:1999mb}.
The code has demonstrated its ability to compute at NLO high multiplicity processes up to $2 \to 7$ \cite{Denner:2015yca,Denner:2016wet}.
In particular the full NLO corrections to VBS and its irreducible background \cite{Biedermann:2016yds,Biedermann:2017bss} have been obtained thanks to this tool.
One key aspect for these high multiplicity processes is the fast integration which is ensured by using similar phase-space mappings to those of Refs.~\cite{Berends:1994pv,Denner:1999gp,Dittmaier:2002ap}. 
In {\sc MoCaNLO+Recola} no approximation is performed neither at LO nor at NLO.
It implies that, also contributions stemming from EW corrections to the interference are computed.
        
  \item {\sc Phantom}~\cite{Ballestrero:2007xq} is a dedicated tree-level Monte Carlo for six parton final states 
  at $\Pp \Pp,\, \Pp\bar{\Pp}$ and $\Pe^+\Pe^-$ colliders at  $\mathcal O(\alpha^6)$ and $\mathcal O(\alphas^2\alpha^4)$ including interferences between the two sets of diagrams.
It employs complete tree-level matrix elements in the complex-mass scheme~\cite{Denner:1999gp,Denner:2005fg,Denner:2006ic} computed via the modular helicity formalism~\cite{Ballestrero:1999md,Ballestrero:1994jn}.
The integration uses a multichannel approach~\cite{Berends:1984gf} and an adaptive strategy~\cite{Lepage:1977sw}.
{\sc Phantom} generates unweighted events at parton level for both the SM and a few instances of beyond the Standard Model (BSM) theories.

  \item The {\sc Powheg-Box}~\cite{Alioli:2010xd,Frixione:2007vw,Nason:2006hfa} is a framework for matching NLO-QCD calculations with parton showers.
It relies on the user providing the matrix elements and Born phase-space, but will automatically construct FKS \cite{Frixione:1995ms} subtraction terms and the phase space for the real emission.
For the VBS processes all matrix elements are being provided by a previous version of {\sc VBFNLO}~\cite{Arnold:2008rz, Arnold:2011wj, Baglio:2014uba} and hence the approximations used in the {\sc Powheg-Box} are similar to those used in {\sc VBFNLO}.

  \item {\sc VBFNLO}~\cite{Arnold:2008rz, Arnold:2011wj, Baglio:2014uba} is a flexible
    parton-level Monte Carlo for processes with EW bosons. It
    allows the calculation of VBS processes at NLO QCD in the VBS
    approximation, with process IDs between 200 and 290. The corresponding
    $s$-channel contributions are available separately as tri-boson processes with
    semi-leptonic decays, with process IDs in the 400 range\ZM{which range is it?}. These can simply
    be added on top of the VBS contribution. Interferences between the two are therefore neglected.
    The usage of leptonic tensors in the calculation, pioneered in
    Ref.~\cite{Jager:2006zc}, thereby leads to a significant speed improvement over
    automatically generated code.  Besides the SM, also a variety of
    new-physics models including anomalous couplings of the Higgs and gauge
    bosons can be simulated.

  \item {\sc Whizard}~\cite{Moretti:2001zz,Kilian:2007gr} is a multi-purpose
      event generator with the LO matrix element generator {\sc O'Mega}. \ZM{ if NLO results for this processes cannot be provided, we should skip what follows, or at least clarify the limitations}
provides FKS subtraction terms for any NLO process, while virtual matrix
elements are provided externally by {\sc
OpenLoops}~\cite{Cascioli:2011va} (alternatively, {\sc Recola}~\cite{Actis:2012qn,Actis:2016mpe}
(\emph{cf.}\ above) can be used as well). {\sc Whizard} allows to simulate a
huge number of BSM models as well, in particular in
the VBS channel in terms of both higher-dimensional operators as well as explicit
resonances.

\end{itemize}

We conclude this section by summarising the characteristics of the various codes in Tab.~\ref{tab:wg1_codes}.
In particular, it is specified whether
\begin{itemize}
    \item all $s$- and $t/u$-channel diagrams are included;
    \item interferences between different diagrams of diagrams ($s/t/u$-channel) are included at LO;
    \item diagrams which do not feature two resonant vector-bosons are included;
    \item the so-called non-factorisable (NF) QCD corrections, \emph{i.e.}\ the corrections where (real or virtual) gluons are exchanged between different quark lines,
        are included;
    \item EW corrections to the interference of order $\mathcal O (\alpha^5\alphas)$ are included.
    These corrections are of the same order as the NLO QCD corrections to the contribution of order $\mathcal O (\alpha^6$) term.
\end{itemize}

\begin{table*}[ht!]
    \footnotesize
    \begin{tabularx}{\textwidth}{c|c|X|X|X|X|X}
        Code  &  $\mathcal O(\alpha^6)$ $s, t, u$  &  $\mathcal O(\alpha^6)$ interf.  &  Non-res.  & NLO &  NF QCD  &  EW corr. to order $\mathcal O(\alphas \alpha^5)$  \\
        \hline
        \hline
        {\sc Bonsay}        &  $t/u$    &  No       &  Yes, virt. No    &  Yes   & No       &  No  \\
        {\sc Powheg}        &  $t/u$    &  No       &  Yes              &  Yes   & No       &  No  \\
        {\sc MG5\_aMC}      &  Yes      &  Yes      &  Yes              &  Yes   & virt. No &  No \\
        {\sc MoCaNLO+Recola}&  Yes      &  Yes      &  Yes              &  Yes   & Yes      &  Yes  \\
        {\sc PHANTOM}       &  Yes      &  Yes      &  Yes              &  No    & -        & - \\
        {\sc VBFNLO}        &  Yes      &  No       &  Yes              &  Yes   & No       &  No  \\
        {\sc Whizard}       &  Yes      &  Yes      &  Yes              &  No    & -        & - \\
    \end{tabularx}
    \caption{\label{tab:wg1_codes} Summary of the different properties of the computer programs employed in the comparison.}
\end{table*}

    \subsection{Input parameters}
        \label{subsec:inputpar}
        The VBS production mechanism is simulated at the LHC with a center-of-mass energy $\sqrt s = 13 \TeV$. 
The five flavour scheme is used and the NNPDF~3.0 parton density~\cite{Ball:2014uwa} with NLO QCD evolution is employed and strong coupling constant $\alphas\left( \MZ \right) = 0.118$.\footnote{The corresponding {\tt lhaid} in LHAPDF6~\cite{Buckley:2014ana} is 260000.} 
Since the employed PDF set has no photonic density, photon-induced processes are not considered.
Initial-state collinear singularities are factorised with the ${\overline{\rm MS}}$ scheme, consistently with what is done in NNPDF.

For the mass and width of the massive particles, the following values are used:
%
\begin{alignat}{2}
                  \Mt   &=  173.21\GeV,       & \quad \quad \quad \Gt &= 0 \GeV,  \nonumber \\
                \MZOS &=  91.1876\GeV,      & \quad \quad \quad \GZOS &= 2.4952\GeV,  \nonumber \\
                \MWOS &=  80.385\GeV,       & \GWOS &= 2.085\GeV,  \nonumber \\
                M_{\rm H} &=  125.0\GeV,       &  \GH   &=  4.07 \times 10^{-3}\GeV,
\end{alignat}
%
and the EW coupling is renormalised in the $G_\mu$ scheme \cite{Denner:2000bj} where
%
\begin{equation}
    G_{\mu}    = 1.16637\times 10^{-5}\GeV^{-2}.
\end{equation}
%
The numerical value of $\alpha$, corresponding to the choice of input parameters, is
%
\begin{equation}
 \alpha = 7.555310522369 \times 10^{-3}.
\end{equation}
%
The complex-mass scheme~\cite{Denner:1999gp,Denner:2005fg} is used throughout to treat unstable intermediate particles in a gauge-invariant manner.

The renormalisation and factorisation scales are set dynamically to
%
\begin{equation}
\label{eq:defscale}
 \mu_{\rm ren} = \mu_{\rm fac} = \sqrt{p_{\rm T, j_1}\, p_{\rm T, j_2}}.
\end{equation}
%
This choice of scale has been shown to provide stable NLO predictions \cite{Denner:2012dz}.

The cross sections and distribution are computed within the following event selecton.
These are inspired from experimental measurements \cite{Aad:2014zda,Aaboud:2016ffv,Khachatryan:2014sta,CMS:2017adb}.

\begin{itemize}
    \item The two same-sign charged leptons are required to have
        \begin{align}
         \ptsub{\Pl} >  20\GeV,\qquad |y_{\Pl}| < 2.5, \qquad \Delta R_{\Pl\Pl}> 0.3\,.
        \end{align}
    \item The total missing transverse energy, computed from the vectorial sum of the transverse momenta of the two neutrinos in the event,
        is required to be
        \begin{align}
          \etsub{\text{miss}}=p_{\rm T, miss} >  40\GeV\,.
        \end{align}
    \item QCD partons (quarks and gluons) are clustered together using the anti-$k_T$ algorithm~\cite{Cacciari:2008gp} with distance parameter $R=0.4$. Jets are required
        to have
        \begin{align}
         \ptsub{\Pj} >  30\GeV, \qquad |y_\Pj| < 4.5, \qquad \Delta R_{\Pj\Pl} > 0.3 \,.
        \end{align}
        On the two jets with largest transverse-momentum the following invariant-mass and rapidity-separation cuts are imposed
        \begin{align}
         m_{\Pj \Pj} >  500\GeV,\qquad |\Delta y_{\Pj \Pj}| > 2.5.
        \end{align}
%         Finally, all jest in the event are required to be separated from charged leptons:
%         \begin{align}
%          \qquad\Delta R_{\Pj\Pl} > 0.3 .
%         \end{align}
    \item When EW corrections are computed, real photons and charged fermion are clustered together using the anti-$k_T$ algorithm with
        radius parameter $R=0.1$. In this case, leptons and quarks mentioned above must be understood as {\it dressed fermions}. Photons
        which are not combined at this step are clustered with QCD partons to form jets as it is described previously.
\end{itemize}


\section{Leading-order study}
    \label{sec:LO}
    \subsection{Three contributions}
        \label{subsec:contributions}
        % At tree level, there are three contributions to the $\PW^+\PW^+$ production in association with two jets: the pure EW component $\mathcal{O}(\alpha^6)$, the interference $\mathcal{O}(\alphas\alpha^5)$, and the QCD background $\mathcal{O}(\alphas^2\alpha^4)$
% \AK{We have mentioned this many times at this point. Perhaps not worth repeating again?}.
In the present section, the cross sections and distributions are obtained without applying the VBS cuts on $m_{\Pj\Pj}$ and $|\Delta y_{\Pj\Pj}|$, 
Eq.~(\ref{cut:4}).
In Tab.~\ref{tab:LOscanXsec}, the cross sections of the three contributions are reported.
The EW, QCD, and interference contributions amount to $57\%$, $37\%$, and $6\%$ of the total inclusive cross section, respectively.
The QCD contribution does not posses external gluons due to charge conservation.
Thus the diagrams of order $\mathcal{O}(\alphas^2\alpha^4)$ only involve gluon exchange in the $t/u$-channel between the quark lines.
This results in a small contribution although the VBS cuts have not been imposed.
The interference between EW and QCD contributions is small, due to color suppression, but not negligible.
% ($t/u$ and $t/s$ interferences with identical fermions).

\begin{table}[h!]
    \centering
    \begin{tabular}{c|c|c|c}
        Order & $\mathcal{O}(\alpha^6)$ & $\mathcal{O}(\alphas^2\alpha^4)$ & $\mathcal{O}(\alphas\alpha^5)$ \\
        \hline
        \hline
        $\sigma[\rm{fb}]$ & $ 2.292 \pm 0.002 $ & $ 1.477 \pm 0.001 $ & $ 0.223 \pm 0.003 $ \\
%         {\sc Xxx}&  $ \pm $ & $ \pm $ & $ \pm $
    \end{tabular}
    \caption{\label{tab:LOscanXsec} Cross sections at LO accuracy for the three contributions to the process ${\rm p}{\rm p}\to\mu^+\nu_\mu{\rm e}^+\nu_{\rm e}{\rm j}{\rm j}$, obtained with exact matrix elements.
    These results are for the set-up described in Sec.~\ref{subsec:inputpar} but no cuts on $m_{\Pj\Pj}$ and $|\Delta y_{\Pj\Pj}|$ are applied.
    The uncertainties shown refer to the estimated statistical error of the Monte Carlo programs.}
\end{table}

In Fig.~\ref{fig:mjjdyjj_1d} these three contributions are shown separately and summed in the differential distribution of the di-jet invariant mass $m_{\Pj\Pj}$ and the rapidity difference $|\Delta y_{\Pj\Pj}|$.
For the distribution in the di-jet invariant mass (left), one can observe that the EW contribution peaks around an invariant mass of about $80\GeV$.
These are due to diagrams where the two jets originate from the decay of a W boson (see middle and right diagrams in Fig.~\ref{diag:LO}).
Note that these contributions are not present in calculations relying on the VBS approximation.
The EW contribution becomes dominant for di-jet invariant mass larger than $500\GeV$.
The same holds true for jet rapidity difference larger than $2.5$ (right).
This justifies why cuts on these two observables are used in order to enhance the EW contribution over the QCD one.
In particular, in order to have a large EW contribution, rather exclusive cuts are required.

\begin{figure*}[hbt]
\centering
\includegraphics[scale=0.395]{figures/scanfigures/mjj_full.pdf}
\includegraphics[scale=0.395]{figures/scanfigures/dyjj_full.pdf}
\caption{Differential distribution in the di-jet invariant mass $m_{\Pj\Pj}$ (left) and the difference of the jet rapidities $|\Delta y_{\Pj\Pj}|$ (right) for the three LO contributions to the process ${\rm p}{\rm p}\to\mu^+\nu_\mu{\rm e}^+\nu_{\rm e}{\rm j}{\rm j}$.
The EW contribution is in red, the QCD one in green, and the interference one in grey.
The sum of all the contributions is in blue.
The cuts applied are the ones of of Sec.~\ref{subsec:inputpar} but no cuts on $m_{\Pj\Pj}$ and $|\Delta y_{\Pj\Pj}|$ are applied.}
\label{fig:mjjdyjj_1d}
\end{figure*}

This can also be seen in Fig.~\ref{fig:mjjdyjj_2d_LO} where the three contributions are displayed as a (double-differential) function of the di-jet invariant mass and jet rapidity difference.
Again, it is obvious that the region with low di-jet invariant mass should be avoided in VBS studies as it is dominated by tri-boson contributions.
% This motivates in particular the choice of $m_{\Pj\Pj} > 200\GeV$ and $|\Delta y_{\Pj\Pj}| > 2$ for our inclusive study (see below).
%Finally, let us notice that the choice $m_{\Pj\Pj} > 500\GeV$ and $|\Delta y_{\Pj\Pj}| > 2.5$ made by the experimental collaborations is well motivated in order to enhance the EW contribution over its irreducible backgrounds.
This motivates in particular the choice of employing the cut $m_{\Pj\Pj} > 200\GeV$ for our LO inclusive study in Sec.~\ref{subsec:LOinclusive}.

\begin{figure*}[hbt]
\centering
\includegraphics[scale=0.395]{figures/scanfigures/scan_ew6.pdf}
\includegraphics[scale=0.395]{figures/scanfigures/scan_ew5qcd1.pdf}
\includegraphics[scale=0.395]{figures/scanfigures/scan_ew4qcd2.pdf}
\caption{Double-differential distributions in the variables $m_{\Pj\Pj}$ and $|\Delta y_{\Pj\Pj}|$ for the three LO contributions of orders $\mathcal{O}(\alpha^6)$ (top left), $\mathcal{O}(\alphas\alpha^5)$ (top right), and $\mathcal{O}(\alphas^2 \alpha^4)$ (bottom).
The cuts applied are the ones of of Sec.~\ref{subsec:inputpar} but no cuts on $m_{\Pj\Pj}$ and $|\Delta y_{\Pj\Pj}|$ are applied.
}
\label{fig:mjjdyjj_2d_LO}
\end{figure*}

    \subsection{Inclusive comparison}
        \label{subsec:LOinclusive}
        In Fig.~\ref{fig:ratio2d_LO}, ratios of double-differential cross sections in the plane $\left(m_{\Pj\Pj}, \Delta y_{\Pj\Pj}\right)$ is shown.\footnote{In Fig.~\ref{fig:ratio2d_LO}, the level of the accuracy of the predictions in each bin is around a per mille.}
Two plots are displayed: the ratios of the $|t|^2 + |u|^2$ and $|s|^2 + |t|^2 + |u|^2$ approximations over the full calculation.
In the first case, the approximation is good within $\pm10\%$ over the whole range apart in the low invariant-mass region at both low and large rapidity difference.
The low rapidity-difference region possesses remnants of the tri-bosons contributions that have a di-jet invariant mass around the $\PW$-boson mass.
It is therefore expected that the $|t|^2 + |u|^2$ approximation fails in this region.
The second plot, where the $|s|^2 + |t|^2 + |u|^2$ approximation is considered, displays a better behaviour in the previously mentioned region.
The full calculation is approximated at the level of $\pm5\%$ apart in the region where $\Delta y_{\Pj\Pj} < 2$.

\begin{figure*}[hbt]
\centering
\includegraphics[scale=0.395]{figures/scanfigures/ratio_tu.pdf}
\includegraphics[scale=0.395]{figures/scanfigures/ratio_stu.pdf}
\caption{Ratio of cross sections per bin in the plan $\left(m_{\Pj\Pj}, |\Delta y_{\Pj\Pj}| \right)$ at LO \emph{i.e.}\ order $\mathcal{O}(\alpha^6)$.
Ratio of approximated squared amplitudes over the full matrix element.
The approximated squared amplitudes are computed as $|\mathcal{A}|^2 \sim |t|^2 + |u|^2$ (left) and $|\mathcal{A}|^2 \sim |s|^2 + |t|^2 + |u|^2$ (right).
The cuts applied are the one of of Sec.~\ref{subsec:inputpar} and no cuts on $m_{\Pj\Pj}$ and $|\Delta y_{\Pj\Pj}|$ are applied.} 
\label{fig:ratio2d_LO}
\end{figure*}
% As explained previously, the low di-jet invariant mass and low jet rapidity separation regions are dominated by tri-boson production.

% Therefore, the \emph{inclusive} study at NLO is only performed in the region 
% %
% \begin{equation}
% \label{eq:inclusive}
% 	m_{jj} > 200\GeV\, \qquad {\rm and} \qquad |\Delta y_{jj}| > 2 .
% \end{equation}
% %
% Hence, the differences arising at NLO in this fiducial region originate solely from NLO effects.

% As explained previously, the low di-jet invariant mass are dominated by tri-boson production.
% Therefore, a comparison of the various approximations in an \emph{inclusive} phase-space volume should exclude the region where tri-boson contributions are dominating.
% To that end, we have chosen for the inclusive fiducial volume to take the following values for the VBS cuts:
%

    \subsection{Comparison in the fiducial region}
        \label{subsec:LOfiducial}
         In Tab.~\ref{tab:wg1_LOrates} we report the total rates at LO accuracy obtained with the set-up described above, and in Fig.~\ref{fig:wg1_mjj-llLO} we show the results
for the tagging-jet (left) and lepton-pair (right) invariant-mass distribution. In both case we show the absolute distributions in the main frame of the 
figures, while in the inset the ratio over {\sc VBFNLO} is displayed. For both observables we find 
an excellent agreement among the various tools, which confirms the fact
that contributions from $s-$channel diagrams as well as from non-resonant configurations are strongly suppressed in the fiducial region. We have checked 
that the same level of agreement holds for many other differential distributions.
\begin{table}[h!]
    \centering
    \begin{tabular}{c|c}
        Code  &  $\sigma[\rm{fb}]$  \\
        \hline
        \hline
        {\sc Bonsay}  &  $X \pm 0.0002$ \\
        {\sc MG5\_aMC}&  $X \pm 0.001$  \\ 
        {\sc MoCaNLO+Recola}  &  $1.4347 \pm 0.0001$ \\
        {\sc PHANTOM}&  $1.4374 \pm 0.0006 $  \\
        {\sc POWHEG}  &  $1.44092 \pm 0.00009$ \\
        {\sc VBFNLO}  &  $1.43796 \pm 0.00005$ \\
        {\sc Whizard}&  $1.4363 \pm 0.0009 $
    \end{tabular}
    \caption{\label{tab:wg1_LOrates} Rates at LO accuracy within VBS cuts obtained with the different codes used in this comparison, 
    for the ${\rm p}{\rm p}\to\mu^+\nu_\mu{\rm e}^+\nu_{\rm e}{\rm j}{\rm j}$ process.}
\end{table}


        In Fig.~\ref{fig:wg1_mjj-llLO}, we show the distributions in the invariant mass (left) and the rapidity difference of the two tagging jets (right) which are key observables for VBS measurements.
In both cases we show the absolute distributions in the upper plot, while the lower plot displays the ratio over the predictions of {\sc MoCaNLO+Recola}, 
for which we also display the scale-uncertainty band (seven-points variation).
For both observables we find a relatively good agreement among the various tools, which confirms the fact that contributions from $s$-channel diagrams as well as interferences are suppressed in the fiducial region.
In general, the agreement is at the level of $1\%$ or below in each bin.
We have checked that the same level of agreement holds for other standard differential distributions such as rapidity, invariant mass, or transverse momentum.
This means that at LO, in the fiducial volume and for energies relevant to the LHC, the VBS approximation is good to a per cent.
This is in agreement with the findings of Sec.~\ref{subsec:LOinclusive} as the present comparison completely excludes the phase-space region where tri-boson contributions could have a noticeable impact.

 \begin{figure*}[htb!]
   \centering
   \includegraphics[width=0.4\textwidth,angle=0,clip=true,trim={0.4cm 2cm 0.cm 1.cm}]{figures/LO/mjj_LO.pdf}
   \includegraphics[width=0.4\textwidth,angle=0,clip=true,trim={0.4cm 2cm 0.cm 1.cm}]{figures/LO/dyj1j2_LO.pdf}
\caption{\label{fig:wg1_mjj-llLO} Differential distributions in the invariant mass (left) and rapidity difference of the two tagging jets (right) at LO accuracy and order $\mathcal{O}(\alpha^6)$.
The description of the different programs used can be found in Sec.~\ref{subsec:codedescr}.
The upper plots provides the absolute value for each prediction while the lower plots presents all predictions normalised to {\sc MoCaNLO}+{\sc Recola} which is one of the full predictions.
The band corresponds to a seven-points scale variation of the renormalisation and factorisation scale.
The predictions are obtained in the fiducial region described in Sec.~\ref{subsec:inputpar}.}
\end{figure*}


\section{Next-to-leading order QCD}
    \label{sec:NLO}
    \subsection{Inclusive comparision}
        \label{subsec:NLOinclusive}
        Scan at NLO

\begin{figure}[hbt]
\centering
\includegraphics[scale=0.39]{figures/scanfigures/a6as_vbfnloVSrecola_stu.pdf}
FIGURE
\caption{Cross section (fb) per bin of $(M_{jj},\,\Delta y_{jj})$ at NLO QCD $\mathcal{O}(\alpha^6\alphas)$, without any cut on the $jj$ pair kinematics:  ratio of approximated squared amplitudes over the full matrix element. The approximated squared amplitudes are computed as $|\mathcal{A}|^2 \sim |s|^2 + |t|^2 + |u|^2$. Results of \texttt{VBFNLO} (approximated) and \texttt{RECOLA} (full) calculations.}\label{fig:mjjdyjj_2d_NLO}
\end{figure}


\begin{figure}[hbt]
\centering
\includegraphics[scale=0.395]{figures/scanfigures/a6as_vbfnloVSrecola_tu.pdf}
%\includegraphics[scale=0.395]{figures/scanfigures/.pdf}
\caption{Cross sections (fb) per bin of $(M_{jj},\,\Delta y_{jj})$ at NLO QCD $\mathcal{O}(\alpha^6\alphas)$, without any cut on the $jj$ pair kinematics: ratio of approximated squared amplitudes over the full matrix element. The approximated squared amplitudes are computed as $|\mathcal{A}|^2 \sim |t|^2 + |u|^2$. Results of \texttt{VBFNLO} (approximated) and \texttt{RECOLA} (full) calculations.}\label{fig:ratio2d_NLO}
\end{figure}
    \subsection{Comparison in the fiducial region}
        Cross sections at NLO

\begin{table}[h!]
    \centering
    \begin{tabular}{c|c|c|c}
        Code  &  $\sigma[\rm{fb}]$  \\
        \hline
        \hline
        {\sc Bonsay}  &  $X \pm 0.0009$  \\
        {\sc MG5\_aMC}&  $X  \pm 0.003$  \\
        {\sc MoCaNLO+Recola}  &  $ 1.382 \pm 0.002$ \\
        {\sc POWHEG}  &  $1.3556 \pm 0.0009$  \\
        {\sc VBFNLO}  &  $1.3916 \pm 0.0001$  \\
    \end{tabular}
    \caption{\label{tab:wg1_NLOrates} Rates at NLO-QCD accuracy within VBS cuts obtained with the different codes used in this comparison, 
    for the ${\rm p}{\rm p}\to\mu^+\nu_\mu{\rm e}^+\nu_{\rm e}{\rm j}{\rm j}$ process.
    \MP{Please add or check your respective numbers}}
\end{table}

At NLO, the rates show slightly larger discrepancies, as it can be observed in Tab.~\ref{tab:wg1_NLOrates}.
This is most likely due to low di-jet invariant-mass configurations, where $s$-channel diagrams and interferences are less suppressed than at LO, because of the presence of extra QCD radiation.
        
In Figs.~(\ref{fig:distNLO1}-\ref{fig:distNLO3}), several differential distributions are shown.
All these predictions are performed at NLO accuracy at the order $\mathcal{O}(\alpha_{\rm s}\alpha^6)$.
\MP{Physics and conclusion on interference/non-factorisable etc.\ effects are not addresses yet in the discussion.}

We start with Fig.~\ref{fig:distNLO1} which displays the invariant mass (left) and the rapidity separation (right) of the two tagging jets.
For high invariant mass, all predictions agree rather well.
On the other hand, for low invariant mass, the hierarchy present at the level of the cross section is here reproduced.
The VBS-approximated predictions ({\sc Bonsay} and the {\sc Powheg-Box}) are lower than the full calculation ({\sc MoCaNLO}+{\sc Recola}).
The full calculation is rather well approximated by the hybrid VBS approximation implemented in {\sc MadGraph5\_aMC\-@NLO}.
Finally, {\sc VBFNLO} which includes as well $s$-channel contributions provides larger predictions at low invariant mass.
For the rapidity difference between the two tagging jets, the hierarchy between the predictions is rather similar.

 \begin{figure*}[hbt!]
   \centering
   \includegraphics[width=0.4\textwidth,angle=0,clip=true,trim={0.4cm 2cm 0.cm 1.cm}]{figures/NLO/mjj_NLO.pdf}
   \includegraphics[width=0.4\textwidth,angle=0,clip=true,trim={0.4cm 2cm 0.cm 1.cm}]{figures/NLO/dyj1j2_NLO.pdf}
\caption{\label{fig:distNLO1} Differential distributions in the invariant mass (left) and rapidity difference of the two tagging jets (right).
The LHC process considered is ${\rm p}{\rm p}\to\mu^+\nu_\mu{\rm e}^+\nu_{\rm e}{\rm j}{\rm j}$ at NLO accuracy and order $\mathcal{O}(\alpha_{\rm s}\alpha^6)$.
The description of the different programs used can be found in Sec.~\ref{subsec:codedescr}.
The upper plots provide the absolute value for each prediction while the lower plots presents all predictions normalised to {\sc MoCaNLO}+{\sc Recola} which is one of the full predictions.
The predictions are obtained in the fiducial region described in Sec.~\ref{subsec:inputpar}.
\MP{MG statistics should be improved and the baseline changed to Recola.}
}
\end{figure*}

Concerning the transverse momentum (left) and rapidity (right) of the hardest jet shown in Fig.~\ref{fig:distNLO2}, the situation is rather different.
While {\sc MadGraph5\_aMC\-@NLO} is very close to the full prediction for low transverse momentum, it is diverging from it at larger transverse momentum.
This is in contrast with {\sc Bonsay} and {\sc Powheg} which approximate the full computation reasonably well over the whole range and in particular in the high transverse-momentum region.
Finally, {\sc VBFNLO} predicts higher rates over the whole range apart from around $200\GeV$ where it is in perfect agreement with the complete calculation.
Concerning the rapidity of the hardest jet, {\sc VBFNLO} is in good agreement with {\sc MoCaNLO}+{\sc Recola} in the rapidity range $|y_{j_1}| < 3$.
For larger rapidity, the other codes constitute a better description of the full process at order $\mathcal{O}(\alpha_{\rm s}\alpha^6)$.

 \begin{figure*}[hbt!]
   \centering
   \includegraphics[width=0.4\textwidth,angle=0,clip=true,trim={0.4cm 2cm 0.cm 1.cm}]{figures/NLO/ptj1_NLO.pdf}
   \includegraphics[width=0.4\textwidth,angle=0,clip=true,trim={0.4cm 2cm 0.cm 1.cm}]{figures/NLO/yj1_NLO.pdf}
\caption{\label{fig:distNLO2} Differential distributions in the transverse momentum (left) and rapidity of the hardest jet (right).
The LHC process considered is ${\rm p}{\rm p}\to\mu^+\nu_\mu{\rm e}^+\nu_{\rm e}{\rm j}{\rm j}$ at NLO accuracy and order $\mathcal{O}(\alpha_{\rm s}\alpha^6)$.
The description of the different programs used can be found in Sec.~\ref{subsec:codedescr}.
The upper plots provide the absolute value for each prediction while the lower plots presents all predictions normalised to {\sc MoCaNLO}+{\sc Recola} which is one of the full predictions.
The predictions are obtained in the fiducial region described in Sec.~\ref{subsec:inputpar}.
\MP{MG statistics should be improved and the baseline changed to Recola.}
}
\end{figure*}

The last set of differential distributions is the invariant mass of the two charged lepton (left) and the Zeppenfeld variable for the anti-muon (right).
Concerning the comparison of the predictions, both distributions display a rather similar behaviour.
Indeed, the hierarchy mentioned previously is here respected and enhanced towards high invariant mass or high Zeppenfeld variable.
{\sc MoCaNLO}+{\sc Recola} and {\sc VBFNLO} are in rather good agreement for both distributions for the kinematic range displayed here.
The other three VBS approximations are close to each other within few per cent.

 \begin{figure*}[hbt!]
   \centering
   \includegraphics[width=0.4\textwidth,angle=0,clip=true,trim={0.4cm 2cm 0.cm 1.cm}]{figures/NLO/mll_NLO.pdf}
   \includegraphics[width=0.4\textwidth,angle=0,clip=true,trim={0.4cm 2cm 0.cm 1.cm}]{figures/NLO/zmu_NLO.pdf}
\caption{\label{fig:distNLO3} Differential distributions in the invariant mass of the two charged leptons (left) and Zeppenfeld variable for the muon (right).
The LHC process considered is ${\rm p}{\rm p}\to\mu^+\nu_\mu{\rm e}^+\nu_{\rm e}{\rm j}{\rm j}$ at NLO accuracy and order $\mathcal{O}(\alpha_{\rm s}\alpha^6)$.
The description of the different programs used can be found in Sec.~\ref{subsec:codedescr}.
The upper plots provide the absolute value for each prediction while the lower plots presents all predictions normalised to {\sc MoCaNLO}+{\sc Recola} which is one of the full predictions.
The predictions are obtained in the fiducial region described in Sec.~\ref{subsec:inputpar}.
\MP{MG statistics should be improved and the baseline changed to Recola.}
}
\end{figure*}


\section{Matching to parton shower}
    \label{sec:matching}
    We now turn to discuss how different predictions compare when the matching to parton-shower (PS) is included. For such
a comparison we expect larger discrepancy than what we found at fixed-order, as a consequence of the different
matching schemes, PS employed and of the other details of the matching (such as the choice of the shower initial scale). Among
the codes capable of providing fixed-order results, presented before, {\sc MG5\_aMC}, {\sc POWHEG} and {\sc VBFNLO}
can also provide results at (N)LO+PS accuracy. Besides, also {\sc PHANTOM} is employed for LO+PS results.\\
{\sc MG5\_aMC},
which
employs the {\sc MC@NLO}~\cite{Frixione:2002ik} matching procedure, will be used together with {\sc Pythia8}~\cite{Sjostrand:2014zea} (version 2.2.3)
and {\sc Herwig++}~\cite{Bahr:2008pv, Bellm:2013hwb} (version 2.7.1). For {\sc POWHEG}, the omonymous matching procedure is 
employed~\cite{Nason:2004rx,Frixione:2007vw}, together with {\sc Pythia8}
{\bf MZ VERSION? if same as MG5, put it at the end together with the tune}. {\sc VBFNLO} makes it possible to choose between the {\sc MC@NLO} and {\sc POWHEG}
matching, in both cases together with {\sc Herwig7}~CITEREF. Finally, {\sc PHANTOM} results will be shown matched with {\sc Pythia 8}.
Whenever {\sc Pythia8}\ is used, the Monash tune~\cite{Skands:2014pea} is selected.\\

Results will be presented within the cuts described in Section~\ref{subsec:inputpar}, applied after shower and hadronization (this implies that jets
are obtained by clustering stable hadrons, and not QCD partons). This implies that at the event-generation level, looser cuts (or no cuts at all)
must be employed in order not to bias the results. {\bf MZ lepton-jet separation at the hard-event level?}.\\

A slightly different setup has been employed for {\sc MG5\_aMC} in order to simplify the calculation: instead of generating the full
${\rm p}{\rm p}\to\mu^+\nu_\mu{\rm e}^+\nu_{\rm e}{\rm j}{\rm j}$ process, since it is anyway dominated by doubly-resonant contribution, the
events are produced for the process with two stable W$^+$ bosons (${\rm p}{\rm p}\to{\rm W^+}{\rm W^+}{\rm j}{\rm j}$), and these W$^+$ bosons
are decayed with {\sc MadSpin}~\cite{Artoisenet:2012st} (keeping spin correlations) before the PS. Since {\sc MadSpin}\ computes
the partial and total decay width of the W bosons at LO accuracy only, while in Section~\ref{subsec:inputpar} the NLO width is employed,
a small effect (6\%) on the normalisation of distribution is induced. Finally, when the renormalisation
and factorisation scales are set, the $\Delta R_{\Pj\Pl}$ cut is not imposed during the jet-clustering procedure, but this has no visible effect
on the results.

We now turn to present the results:



\begin{table}[h!]
    \centering
    \begin{tabular}{c|c|c|c}
        Code  &  $\sigma[\rm{fb}]$  \\
        \hline
        {\sc MG5\_aMC}+{\sc Pythia8}&  $1.450 (1.368) \pm 0.$  \\
        {\sc MG5\_aMC}+{\sc Herwig++}&  $1.445 (1.363) \pm 0.$  \\
        {\sc POWHEG}  &  $1.3633 \pm 0.0004$  \\
        {\sc VBFNLO}  &  $1.339 \pm 0.$  \\
        \hline
        {\sc MG5\_aMC}+{\sc Pythia8} (LO)&  $1.352 (1.275) \pm 0.$  \\
        {\sc MG5\_aMC}+{\sc Herwig++} (LO)&  $1.343 (1.267)  \pm 0.$  \\
        {\sc PHANTOM}+{\sc Pythia8} &  $1.235  \pm 0.$  \\
    \end{tabular}
    \caption{\label{tab:wg1_NLOrates} Rates at NLO-QCD (LO-QCD) accuracy matched to parton shower within VBS cuts obtained with the different codes used in this comparison,
    for the ${\rm p}{\rm p}\to\mu^+\nu_\mu{\rm e}^+\nu_{\rm e}{\rm j}{\rm j}$ process. Numbers in parentheses for the {\sc MG5\_aMC} simulations
    are rescaled to account for the effect related to the boson widths computed by {\sc MadSpin}, see the text for details.}
{\bf MZ: MC uncertainties???}
\end{table}

\begin{figure*}[hbt]
\centering
\includegraphics[width=0.47\textwidth]{figures/LOPS/jetsexclusive.pdf}
\includegraphics[width=0.47\textwidth]{figures/NLOPS/jetsexclusive.pdf}
\caption{Exclusive jet multiplicity from predictions matched to parton shower, at LO (left) or NLO (right) accuracy, compared with the fixed-NLO result
    computed with \sc{VBFNLO}}
\label{fig:PSnjet}
\end{figure*}

\begin{figure*}[hbt]
\centering
\includegraphics[width=0.47\textwidth]{figures/LOPS/m_jj.pdf}
\includegraphics[width=0.47\textwidth]{figures/NLOPS/m_jj.pdf}
\caption{Same as in Fig.~\protect\ref{fig:PSnjet}, for the invariant mass of the two tagging jets.}
\label{fig:PSmjj}
\end{figure*}

\begin{figure*}[hbt]
\centering
\includegraphics[width=0.47\textwidth]{figures/LOPS/Deltay_jj.pdf}
\includegraphics[width=0.47\textwidth]{figures/NLOPS/Deltay_jj.pdf}
\caption{Same as in Fig.~\protect\ref{fig:PSnjet}, for the rapidity separation of the two tagging jets.}
\label{fig:PSdyjj}
\end{figure*}

\begin{figure*}[hbt]
\centering
\includegraphics[width=0.47\textwidth]{figures/LOPS/pT_j1.pdf}
\includegraphics[width=0.47\textwidth]{figures/NLOPS/pT_j1.pdf}
\caption{Same as in Fig.~\protect\ref{fig:PSnjet}, for the transverse momentum of the hardest jet.}
\label{fig:PSpt1}
\end{figure*}

\begin{figure*}[hbt]
\centering
\includegraphics[width=0.47\textwidth]{figures/LOPS/pT_j2.pdf}
\includegraphics[width=0.47\textwidth]{figures/NLOPS/pT_j2.pdf}
\caption{Same as in Fig.~\protect\ref{fig:PSnjet}, for the transverse momentum of the second-hardest jet.}
\label{fig:PSpt2}
\end{figure*}

\begin{figure*}[hbt]
\centering
\includegraphics[width=0.47\textwidth]{figures/LOPS/y_j2.pdf}
\includegraphics[width=0.47\textwidth]{figures/NLOPS/y_j2.pdf}
\caption{Same as in Fig.~\protect\ref{fig:PSnjet}, for the rapidity of the second-hardest jet.}
\label{fig:PSy2}
\end{figure*}

\begin{figure*}[hbt]
\centering
\includegraphics[width=0.47\textwidth]{figures/LOPS/y_j3.pdf}
\includegraphics[width=0.47\textwidth]{figures/NLOPS/y_j3.pdf}
\caption{Same as in Fig.~\protect\ref{fig:PSnjet}, for the rapidity of the third-hardest jet.}
\label{fig:PSy3}
\end{figure*}

\begin{figure*}[hbt]
\centering
\includegraphics[width=0.47\textwidth]{figures/LOPS/z_j3.pdf}
\includegraphics[width=0.47\textwidth]{figures/NLOPS/z_j3.pdf}
\caption{Same as in Fig.~\protect\ref{fig:PSnjet}, for the $z$ variable of the third-hardest jet.}
\label{fig:PSz3}
\end{figure*}



\section{Conclusion}
\label{sec:conclusion}

In the present article, a detailed study of the process $\Pp\Pp\to\mu^+\nu_\mu{\rm e}^+\nu_{\rm e}\,\Pj\Pj+\mathrm{X}$ at the LHC has been presented.
The main focus is the electroweak (EW) production of such a final state where vector-boson (VBS) occurs.
So far, all NLO computations have been performed in the so-called VBS approximation.
Only recently a full computation became available \cite{Biedermann:2017bss}.
Therefore, various theoretical predictions have been compared to the full computation.
This has not only been performed in a typical VBS fiducial region but also in more inclusive volumes.
We quantify precisely the differences that arise for several physical observables and in particular for the di-jet invariant mass and the rapidity-separation of the leading two jets.
This is the first time that such a in-depth study is performed and should be representative of the quality of the approximations for other VBS signatures.
In addition to fixed order predictions we have also investigated the impact of parton-shower.
To that end, several fixed-order computations matched to parton shower have been used.
It turns out that very large differences can appear for certain observables in the central region.
These differences appear in the central region where for VBS, colour-recombination plays a significant role.
These finding are new and should trigger further investigations in the theoretical community as well as by experimental collaborations.
Indeed, measurements of such observables should allow for a better understanding of colour-recombination in parton shower.
\MP{It might be too strong a statement.}
The results presented here are exclusively theoretical.
Nonetheless, they should raise significant interests in the experimental collaborations.
Therefore, to supplement this summary, we provide several recommendations for the use of theoretical predictions for the quest of measuring VBS processes precisely.


\MP{This might deserve a section on its own.} \\
Recommendations to experimental collaborations: \\
- Stress that this is only for W+W+ but that qualitative results should apply to other VBS signatures with massive gauge bosons.
- Combinations with EW NLO corrections. \\
- Missing higher EW order: $\pm \delta^2_{\rm NLO EW}$ \\
- Systematics when using NLO QCD approximation \\
- Systematics of different parton shower \\
- Combined measurement including  EW, QCD, and interference \\
- Move to NLO predictions / generators \\
- Comment on the irreducible QCD background \\
- Uncertainties related to PDF.
Some of us have already presented preliminary results on this subjetc \cite{}. % Radcor proceedings from Christopher.
A forthcoming article will addresses related questions.


\section*{Acknowledgements}

We would like to thank \ldots \\
the {\sc Pythia8} authors, in particular Stefan Prestel, Torbjorn Sjostrand and Peter Skands for 
discussions and clarifications about the third-jet rapidity spectrum. 

The authors would like to acknowledge the contribution of the COST Action CA16108 which initiated this work.
Moreover, this work was supported by sveral STSM Grant from the COST Action CA16108.

BB, AD, and MP acknowledge financial support by the
German Federal Ministry for Education and Research (BMBF) under
contract no.~05H15WWCA1 and the German Science Foundation (DFG) under
reference number DE 623/6-1.

AK acknowledge financial support by the Swiss National Science Foundation (SNF) under contract 200020-175595


% \appendix
% 
% \section{Appendix one}


\bibliographystyle{utphys}
\bibliography{article}

\end{document}
