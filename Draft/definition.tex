The scattering of two positively charge W bosons is proceeding at the LHC through the partonic process:
%
\begin{equation}
 {\rm p}{\rm p}\to\mu^+\nu_\mu{\rm e}^+\nu_{\rm e}{\rm j}{\rm j}.
\end{equation}

This process possesses three LO contributions of different orders.
The first one is of order $\mathcal{O}{\left(\alpha^{6}\right)}$ is referred to as EW contributions.
In addition to typical VBS contributions as shown on the left of Fig.~\ref{diag:LO}, it also features $s$-channel contributions.
These type of contributions will play a particular role in the study of the various contributions in Section \ref{subsec:contributions}.
Some of them take the form of decay chains as for example the diagram represented in the middle of Fig.~\ref{diag:LO} while others are tri-boson contributions (right of Fig.~\ref{diag:LO})
The VBS diagrams typically dominate the full process.
But all these contributions form a single gauge-invariant set of contributions and therefore cannot be separated.

The process can also be mediated via a gluon connecting the two quark lines while the W bosons are radiated from the quark lines.
These contributions are of order $\mathcal{O}{\left(\alphas^{2}\alpha^{4}\right)}$ and usually feature different kinematic behaviours than the EW contribution.
Nonetheless they share the same final state and therefore constitute an irreducible background.

Finally, due to the specific colour structure of these two classes of amplitudes, there exists non-zero interferences.
These are of order $\mathcal{O}{\left(\alpha_{\rm s}\alpha^{5}\right)}$ and are usually small but not negligible for realistic experimental set-ups \cite{Biedermann:2017bss}.

Usually, in experimental measurements, special VBS-cuts are designed in order to enhance the EW contribution over the QCD one.
These cuts are based on the fact that the two contributions have rather different kinematic.
The EW contribution is characterised by two jets with large rapidity in the peripheral region as well as a large invariant mass.
The two W bosons are mostly produced centrally.
This is in opposition with the QCD component which favour jets in the central region.
Therefore, the event selection usually involve rapidity-difference and invariance mass cuts for the jets.
This will also be discussed in Section \ref{subsec:contributions}.

\begin{figure*}[t]
\begin{center}
          \includegraphics[width=0.30\linewidth]{feynman/LO_EW_5}
          \raisebox{.5ex}{\includegraphics[width=0.35\linewidth]{feynman/LO_EW_2}}
          \raisebox{-1.8ex}{\includegraphics[width=0.32\linewidth]{feynman/LO_EW_3}}
\end{center}
        \caption{Sample tree-level diagrams that contribute to the process $\Pp\Pp\to\mu^+\nu_\mu\Pe^+\nu_{\Pe}\Pj\Pj$ at order $\mathcal{O}{\left(\alpha^{6}\right)}$.
        In addition to typical VBS contribution (left), this order also possesses $s$-channel (middle) and tri-boson contributions (right). }
\label{diag:LO}
\end{figure*}
