In Tab.~\ref{tab:wg1_NLOrates}, the cross sections of the various tools at NLO-QCD accuracy are presented.
The order considered is again the order $\mathcal{O}(\alpha_{\rm s}\alpha^6)$ and the fiducial volume is the one described in Sec.~\ref{subsec:inputpar}.
In contrast with Tab.~\ref{tab:wg1_LOrates}, the NLO predictions differ visibly according to the approximations used.

\begin{table}[h!]
    \centering
    \begin{tabular}{c|r@{ $\pm$ }l}
      Code  &  \multicolumn{2}{c}{$\sigma[\rm{fb}]$}  \\
        \hline
        \hline
        {\sc Bonsay}  &  $X$ & $0.0009$  \\
        {\sc MG5\_aMC}&  $1.363\phantom{0}$ & $0.004$  \\
        {\sc MoCaNLO+Recola}  &  $ 1.382\phantom{0}$ & $0.002$ \\
        {\sc Powheg-Box}  &  $1.362$   $0.003$   \\
        {\sc VBFNLO}  &  $1.3916$ & $0.0001$  \\
    \end{tabular}
    \caption{\label{tab:wg1_NLOrates} Cross sections for the LHC process ${\rm p}{\rm p}\to\mu^+\nu_\mu{\rm e}^+\nu_{\rm e}{\rm j}{\rm j}$ at NLO accuracy and order $\mathcal{O}(\alpha^6)$.
    The uncertainties shown refers to estimated statistical error of the Monte Carlo programs.
    The predictions are obtained in the fiducial region described in Sec.~\ref{subsec:inputpar}.
    \MP{Please add or check your respective numbers.}}
    \MR{My $t$-/$u$-channel-only number is $1.3703(1)$, so something additionally must be going on with Powheg, which we should comment on.}
    \AK{The number here was wrong. I have updated the table and plots. I am also running some more statistics... }
\end{table}

The first observation is that the predictions featuring two versions of the VBS approximation ({\sc Bonsay} and the {\sc Powheg-Box}) are close.
This means that the double-pole approximation on the two W bosons used in {\sc Bonsay} constitutes a good approximation of the VBS-approximated virtual corrections implemented in the {\sc Powheg-Box}.
Both predictions differ by about $2\%$ with respect to the full computation ({\sc MoCaNLO+Recola}).
The second observation is that the inclusion of $s$-channel contributions seems to have a significant impact.
Indeed, its inclusion (as done in {\sc VBFNLO}) approximates the full computation by less than a per-cent ($0.7\%$).
The main source of the $s$-channel diagrams thereby consists of
real-emission contributions, where one of the two leading jets is formed
by one quark, or possibly also both quarks, originating from the $W$
decay, and the second one by the extra radiation emitted from the
initial state. In such configurations, the hadronically decaying W boson
can become on-shell and hence yield larger contributions than at LO, where the
invariant mass cut on the two jets forces the boson into the far
off-shell region.
