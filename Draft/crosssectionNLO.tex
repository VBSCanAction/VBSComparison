In Tab.~\ref{tab:wg1_NLOrates}, the cross sections of the various tools at NLO-QCD accuracy are presented.
The order considered is again the order $\mathcal{O}(\alpha_{\rm s}\alpha^6)$ and the fiducial volume is the one described in Sec.~\ref{subsec:inputpar}.
In contrast with Tab.~\ref{tab:wg1_LOrates}, the NLO predictions differ visibly according to the approximations used.

\begin{table}[h!]
    \centering
    \begin{tabular}{c|r@{ $\pm$ }l}
      Code  &  \multicolumn{2}{c}{$\sigma[\rm{fb}]$}  \\
        \hline
        \hline
        {\sc Bonsay}  &  $1.35039$ & $0.00006$  \\
        {\sc Powheg-Box}  &  $1.3605\phantom{0}$  & $0.0007$   \\
        {\sc MG5\_aMC}&  $1.363\phantom{0}\phantom{0}$ & $0.004$  \\
        {\sc MoCaNLO+Recola}  &  $ 1.378\phantom{0}\phantom{0}$ & $0.001$ \\
        {\sc VBFNLO}  &  $1.3916\phantom{0}$ & $0.0001$  \\
    \end{tabular}
    \caption{\label{tab:wg1_NLOrates} Cross sections for the LHC process ${\rm p}{\rm p}\to\mu^+\nu_\mu{\rm e}^+\nu_{\rm e}{\rm j}{\rm j}$ at NLO accuracy and order $\mathcal{O}(\alphas\alpha^6)$.
    The uncertainties shown refers to estimated statistical error of the Monte Carlo programs.
    The predictions are obtained in the fiducial region described in Sec.~\ref{subsec:inputpar}.}
\end{table}

The first observation is that the predictions featuring two versions
of the VBS approximation ({\sc Bonsay} and the {\sc Powheg-Box}) are
relatively close.\footnote{The {\sc VBFNLO}-predictions omitting
$s$-channel contributions amounts to $1.3703(1)$ fb. This differs from
the {\sc Powheg-Box} prediction mainly due to the different choice of
scales used in the {\sc Powheg-Box} (\emph{cf.}
footnote \ref{foot:powheg}).} {\sc Bonsay} uses the double-pole
approximation for the virtual matrix element, and it is worth noticing
that this approximation seems to be accurate at $1\%$ level as
compared to the {\sc Powheg-Box}. This means that the double-pole
approximation on the two W bosons used in {\sc Bonsay} constitutes a
good approximation of the VBS-approxi\-mated virtual corrections
implemented in the {\sc Powheg-Box}.  Both predictions differ by about
$2\%$ with respect to the full computation ({\sc MoCaNLO+Recola}).
The second observation is that the inclusion of $s$-channel
contributions seems to have a significant impact.  Indeed, its
inclusion (as done in {\sc VBFNLO}) approximates the full computation
by less than a per-cent ($0.7\%$). The main source of
the $s$-channel diagrams thereby consists of real-emission
contributions, where one of the two leading jets is formed by one
quark, or possibly also both quarks, originating from the W-boson decay,
and the second one by the extra radiation emitted from the initial
state. In such configurations, the hadronically decaying W boson can
become on-shell and hence yield larger contributions than at LO, where
the invariant mass cut on the two jets forces the boson into the far
off-shell region.
However, the agreement between {\sc MoCaNLO+Recola} and {\sc VBFNLO} is mostly accidental, as the inclusion of interference effect and some non-factorisable corrections (in the real corrections) in {\sc MG5\_aMC} brings the prediction down and closer to the VBS approximation.
Not unexpectedly none of the approximations used here agree perfectly with the full calculation of {\sc MoCaNLO\-+Recola} at NLO.
Nevertheless, the disagreement seems to never exceed $2\%$ at the fiducial cross section level.

